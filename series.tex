\section{Chapter 4 Series}
\begin{definition*}{幂级数}
    \[\sum_{n=0}^{+\infty}C_n(z-z_0)^n\]
\end{definition*}
\begin{definition*}{Fourier级数}
    \[\sum_{n=0}^{+\infty}C_ne^{in\theta} = \sum_{n=0}^{+\infty}(a_n\cos n\theta + b_n\sin n\theta)\]
\end{definition*}
\begin{definition*}{Taylor级数}
    \[\sum_{n=0}^{+\infty}\frac{f^{(n)}(z_0)}{n!}(z-z_0)^n\]
\end{definition*}
\begin{definition*}{Laurent级数}
    \[\sum_{n=-\infty}^{+\infty}C_n(z-z_0)^n\]
\end{definition*}
\begin{theorem*}{Abel 定理}
    若$f(z)=\sum_{n=0}^{+\infty}C_nz^n$在$z_0$收敛,则$\forall z$有$|z|<|z_0|$时$f(z)$绝对收敛。
    若存在$z_0$,$f(z)$在$z_0$发散,则$\forall z$有$|z|>|z_0|$时$f(z)$发散。(即幂级数的收敛域是圆盘)
\end{theorem*}
\begin{definition*}{收敛半径}
    若存在常数$R>0$,当$|z|<R$时,$f(z)$绝对收敛,而当$|z|>R$时,$f(z)$发散,这时$R$称为$f(z)$的收敛半径。
\end{definition*}
\begin{theorem*}
    若\[\lim_{n\rightarrow+\infty}\left|\frac{C_n}{C_{n+1}}\right|=\lambda\]
    则$R=\lambda$
\end{theorem*}
\begin{theorem*}
    若\[\lim_{n\rightarrow+\infty} \left|\frac{1}{\sqrt[n]{C_n}}\right|=\lambda\]
    则$R=\lambda$
\end{theorem*}
\begin{theorem*}
    若$f(z)$只有有限个奇点,则离原点最近的奇点$z_0$的模即为收敛半径。
\end{theorem*}
\begin{theorem*}
    若$f(z)$在$z_0$处条件收敛,则$R=|z_0|$
\end{theorem*}
\begin{theorem*}
    若$f(z)=\sum_{n=0}^{+\infty}C_nz^n$满足$C_n = a_n + ib_n\qquad a_n,b_n\in\mathbb{R}$且$\sum_{n=0}^{+\infty}a_nz^n$的收敛半径是$R_1$,$\sum_{n=0}^{+\infty}b_nz^n$的收敛半径是$R_2$,则$R=\mathrm{min}\{R_1,R_2\}$
\end{theorem*}
\begin{theorem*}
当$|z|<R$时,\[f(z)=\sum_{n=0}^{+\infty}\frac{f^{(n)}(0)}{n!}z^n\]
即在收敛圆内,$f(z)$处处满足Cauchy-Riemann条件。根据Abel定理,收敛圆上处处是奇点。
\end{theorem*}
\begin{homeworkProblem}
举出级数在其收敛圆上处处发散、既有发散的点也有收敛的点、处处收敛的例子。
\solution
\begin{enumerate}
    \item 考察
    \[f(z) = \sum_{n=0}^{+\infty}z^n = \frac{1}{1-z}\qquad R=1\]
    $\forall z, |z|=1$,$f(z)$不存在,即收敛圆上处处发散。
    \item 考察
    \[f(z) = \sum_{n=1}^{+\infty}\frac{z^n}{n} \qquad R=1\]
    $f(-1)=-\ln2$但$f(1)=+\infty$发散。更一般的,对$z=e^{i\theta}\quad\theta\in[0,2\pi)$有
    \[\begin{split}
    f(e^{i\theta}) &= \sum_{n=1}^{+\infty}\frac{\cos n\theta}{n}+i\sum_{n=1}^{+\infty}\frac{\sin n\theta}{n} \\
    &= \frac{1}{2}\ln\frac{1}{2(1-\cos\theta)}+i\frac{\pi-\theta}{2}
    \end{split}
    \]
    即收敛圆上除$z=1$外都收敛。
    \item 考察
    \[
    f(z) = \sum_{n=1}^{+\infty}\frac{z^n}{n^2}\qquad R=1
    \]
    因为
    \[
    \sum_{n=1}^{+\infty}\left|\frac{z^n}{n^2}\right| \leq \sum_{n=1}^{+\infty} \frac{1}{n^2} = \frac{\pi^2}{6} < +\infty
    \]
    所以
    \[\forall z:|z|\leq1\]有$f(z)$绝对收敛
\end{enumerate}
\end{homeworkProblem}
\begin{homeworkProblem}
    将$\frac{1}{z-b}$在$z_0=a$处展成Laurent级数,$a\neq b$\\
\solution
\[\begin{split}
\frac{1}{z-b}
&= \frac{1}{-(b-a) + (z-a)}\\
&= \frac{1}{a-b}\frac{1}{1-\frac{z-a}{b-a}} \\
&= \frac{1}{a-b}\sum_{n=0}^{+\infty}\frac{(z-a)^n}{(b-a)^n}\\
&= \frac{1}{a-b}\sum_{n=0}^{+\infty}\frac{(-1)(z-a)^n}{(b-a)^{n+1}}
\end{split}\]
条件
\[
|\frac{z-a}{b-a}| < 1
\]
即\[0\leq|z-a|<|b-a|\]
\end{homeworkProblem}
\begin{homeworkProblem}
    求\[f(z)=\frac{1}{(z-1)(z-2)}\]的Laurent级数\newline
\solution
\begin{enumerate}
    \item 当$0<|z-1|<1$时
    \[\begin{split}
    f(z) &= \frac{(z-1) - (z-2)}{(z-1)(z-2)}\\
    &= \frac{1}{z-2} - \frac{1}{z-1} \\
    &= \frac{1}{z-1 -1} - \frac{1}{z-1} \\
    &= -\frac{1}{1-(z-1)}- \frac{1}{z-1}\\
    &= -\sum_{n=0}^{+\infty}(z-1)^n - \frac{1}{z-1}\\
    &= \sum_{n=-1}^{+\infty}-(z-1)^n \qquad 0<|z-1|<1
    \end{split}\]
    \item 当$|z-1| > 1$时
    \[\begin{split}
    f(z) &= \frac{1}{z-2} - \frac{1}{z-1}\\
    &= \frac{1}{z-1 - 1}- \frac{1}{z-1}\\
    &= \frac{1}{z-1}\frac{1}{1-\frac{1}{z-1}}- \frac{1}{z-1}\\
    &= \frac{1}{z-1}\sum_{n=0}^{+\infty}(\frac{1}{z-1})^n - \frac{1}{z-1}\\
    &=\sum_{n=0}^{+\infty}(\frac{1}{z-1})^{n+1} - \frac{1}{z-1}\\
    &=\sum_{n=1}^{+\infty}(\frac{1}{z-1})^{n+1}\\
    &= \sum_{n=-\infty}^{-2}(\frac{1}{z-1})^{n}
    \end{split}\]
    \item 在$z=2$处展开同理
\end{enumerate}
\end{homeworkProblem}
\newpage
