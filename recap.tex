\documentclass{ctexrep}

\usepackage{fancyhdr}
\usepackage{extramarks}
\usepackage{esint}
\usepackage{amsmath}
\usepackage{amsthm}
\usepackage{amsfonts}
\usepackage{tikz}
\usepackage[plain]{algorithm}
\usepackage{algpseudocode}

\usetikzlibrary{automata,positioning}

%
% Basic Document Settings
%

\topmargin=-0.45in
\evensidemargin=0in
\oddsidemargin=0in
\textwidth=6.5in
\textheight=9.0in
\headsep=0.25in

\linespread{1.1}

\pagestyle{fancy}
\lhead{\hmwkAuthorName}
\chead{\hmwkClass\ : \hmwkTitle}
\rhead{\firstxmark}
\lfoot{\lastxmark}
\cfoot{\thepage}

\renewcommand\headrulewidth{0.4pt}
\renewcommand\footrulewidth{0.4pt}

\setlength\parindent{0pt}

%
% Create Problem Sections
%

\newcommand{\enterProblemHeader}[1]{
    \nobreak\extramarks{}{Problem \arabic{#1} continued on next page\ldots}\nobreak{}
    \nobreak\extramarks{Problem \arabic{#1} (continued)}{Problem \arabic{#1} continued on next page\ldots}\nobreak{}
}

\newcommand{\exitProblemHeader}[1]{
    \nobreak\extramarks{Problem \arabic{#1} (continued)}{Problem \arabic{#1} continued on next page\ldots}\nobreak{}
    \stepcounter{#1}
    \nobreak\extramarks{Problem \arabic{#1}}{}\nobreak{}
}

\setcounter{secnumdepth}{0}
\newcounter{partCounter}
\newcounter{homeworkProblemCounter}
\setcounter{homeworkProblemCounter}{1}
\nobreak\extramarks{Problem \arabic{homeworkProblemCounter}}{}\nobreak{}

%
% Homework Problem Environment
%
% This environment takes an optional argument. When given, it will adjust the
% problem counter. This is useful for when the problems given for your
% assignment aren't sequential. See the last 3 problems of this template for an
% example.
%
\newenvironment{homeworkProblem}[1][-1]{
    \ifnum#1>0
        \setcounter{homeworkProblemCounter}{#1}
    \fi
    \subsection{Problem \arabic{homeworkProblemCounter}}
    \setcounter{partCounter}{1}
    \enterProblemHeader{homeworkProblemCounter}
}{
    \exitProblemHeader{homeworkProblemCounter}
}

%
% Homework Details
%   - Title
%   - Due date
%   - Class
%   - Section/Time
%   - Instructor
%   - Author
%

\newcommand{\hmwkTitle}{Recap}
\newcommand{\hmwkClass}{Introduction to Complex Analysis}
\newcommand{\hmwkAuthorName}{\textbf{Xavier Yao}}

%
% Title Page
%

\title{
    \vspace{2in}
    \textmd{\textbf{\hmwkClass:\ \hmwkTitle}}\\
    \normalsize\vspace{0.1in}\small{Due\ on\ \hmwkDueDate\ at 3:10pm}\\
    \vspace{0.1in}\large{\textit{\hmwkClassInstructor\ \hmwkClassTime}}
    \vspace{3in}
}

\author{\hmwkAuthorName}
\date{}

\renewcommand{\part}[1]{\textbf{\large Part \Alph{partCounter}}\stepcounter{partCounter}\\}

%
% Various Helper Commands
%

% Useful for algorithms
\newcommand{\alg}[1]{\textsc{\bfseries \footnotesize #1}}

% For derivatives
\newcommand{\deriv}[1]{\frac{\mathrm{d}}{\mathrm{d}x} (#1)}

% For partial derivatives
\newcommand{\pderiv}[2]{\frac{\partial}{\partial #1} (#2)}

% Integral dx
\newcommand{\dx}{\mathrm{d}x}

% Alias for the Solution section header
\newcommand{\solution}{\textbf{\large Solution}\newline}

% Probability commands: Expectation, Variance, Covariance, Bias
\newcommand{\E}{\mathrm{E}}
\newcommand{\Var}{\mathrm{Var}}
\newcommand{\Cov}{\mathrm{Cov}}
\newcommand{\Bias}{\mathrm{Bias}}

% Complex analysis
\newcommand{\Res}{\mathrm{Res}}
\newcommand{\Im}{\mathrm{Im}}
\newcommand{\Ln}{\mathrm{Ln}}
\newcommand{\arg}{\mathrm{arg}}
\newcommand{\dtheta}{\mathrm{d}\theta}

% Theorem
\theoremstyle{plain}
\newtheorem*{theorem*}{定理}
\newtheorem*{definition*}{定义}

\begin{document}
\section{Chapter 1 Introduction}
\begin{homeworkProblem}
求\[\mathrm{max}|\alpha z^n+\beta| \qquad |z|\leq r\]
\solution
\[\mathrm{max}|\alpha z^n+\beta| = |\alpha|\mathrm{max}|z^n+\frac{\beta}{\alpha}|\]
因此下面只讨论\[\mathrm{max}|z^n+\alpha| \qquad |z|\leq r\]
有
\[|z^n+\alpha| \leq |z^n| + |\alpha| \leq r^n + |\alpha|\]
取最大值$\Leftrightarrow$等号成立$\Leftrightarrow z^n\textrm{与}\alpha\textrm{同向}\Leftrightarrow z^n = \lambda\alpha\quad \lambda > 0$\\
当$\alpha=0$时
\begin{gather*}
    |z^n| = |z|^n = r^n\\
    |z| = r\\
    \mathrm{max}|\alpha z^n+\beta| = r^n \qquad z=re^{i\thata}\quad \theta\in[0,2\pi)
\end{gather*}
当$\alpha\neq 0$时\[\mathrm{max}|z^n+\alpha|=r^n+|\alpha|\]
此时
\begin{gather*}
r^n = |z^n| = |\lambda\alpha|\\
\lambda = \frac{|r|^n}{|\alpha|}\\
z^n = \lambda\alpha=\frac{|r|^n}{|\alpha|}\alpha
\end{gather*}
设$\alpha = r_0e^{i\theta_0}$
\[\begin{split}
z^n &=\frac{|r|^n}{|\alpha|}\alpha \\
&= \frac{|r|^n}{|\alpha|}r_0e^{i\theta_0} \\
&= |r|^ne^{i\theta_0}
\end{split}\]
则取最大值时\[z=z_k=re^{\frac{i(\theta_0+2k\pi)}{n}} \qquad k=1,2,\cdots,n\]
综上,
\[
\mathrm{max}|z^n+\alpha| = \left\{\begin{array}{lll}
r^n & z=re^{i\theta}\quad \theta\in[0,2\pi) & \alpha=0\\
r^n + |\alpha| & z=re^{\frac{i(\theta_0+2k\pi)}{n}} \quad k=1,2,\cdots,n & \alpha\neq0
\end{array}\right
\]
\end{homeworkProblem}

\begin{homeworkProblem}
    \[|z|^2=z\overline{z}\qquad z^2=|z|^2\qquad\]
等号成立的条件是?\newline
\solution
\[z^2 = |z|^2 = z\overline{z} \Leftrightarrow z(z-\overline{z})=0\Leftrightarrow z=\overline{z}\]
即$z\in\mathbb{R}$时等号成立。
\end{homeworkProblem}


\begin{homeworkProblem}
证明\[|z_1+ z_2|^2 + |z_1 - z_2|^2 = 2(|z_1|^2 + |z_2|^2)\] 并说明几何意义\newline
\begin{proof}
\[\begin{split}
|z_1+ z_2|^2 + |z_1 - z_2|^2
&= (z_1+z_2)\overline{(z_1+z_2)} + (z_1-z_2)\overline{(z_1-z_2)}\\
&=(z_1+z_2)(\overline{z_1}+\overline{z_2}) + (z_1-z_2)(\overline{z_1}-\overline{z_2})\\
&=z_1\overline{z_1} + z_1\overline{z_2} + z_2\overline{z_1} + z_2\overline{z_2} + z_1\overline{z_1}-z_1\overline{z_2}-z_2\overline{z_1} + z_2\overline{z_2} \\
&= 2(z_1\overline{z_1} + z_2\overline{z_2})\\
&= 2(|z_1|^2 + |z_2|^2)
\end{split}\]
\end{proof}
几何意义:平行四边形对角线平方和等于对边平方和
\end{homeworkProblem}

\begin{homeworkProblem}
    $|z_1|=|z_2|=|z_3|=|z_4|=r\textrm{且}z_1+z_2+z_3+z_4=0$则$z_1$,$ z_2$,$z_3$,$z_4$满足什么条件时$z_1z_2z_3z_4$构成正方形?\newline
\solution
\begin{theorem*}
    $z_1,z_2,\cdots,z_n$将圆$|z|=\alpha$n等分$\Leftrightarrow$$z_k$是分圆多项式$z^n+\alpha=0$
\end{theorem*}
当$z_1$,$ z_2$,$z_3$,$z_4$构成正方形时,$(z-z_1)(z-z_2)(z-z_3)(z-z_4)$是分圆多项式。\\
又
\[\begin{split}
&(z-z_1)(z-z_2)(z-z_3)(z-z_4)\\
=& z^4 - \sum_{k=1}^{4}z_kz^3 + (z_1z_2 + z_1z_3 + z_1z_4 + z_2z_3 + z_2z_4+z_3z_4) z^2 \\
&-(z_1z_2z_3 + z_1z_2z_4+z_1z_3z_4+z_2z_3z_4)z + z_1z_2z_3z_4
\end{split}\]
$(z-z_1)(z-z_2)(z-z_3)(z-z_4)${是分圆多项式}
\[\begin{split}
\Leftrightarrow& \left\{\begin{array}{ll}
\sum_{k=1}^{4}z_k &= 0\\
z_1z_2 + z_1z_3 + z_1z_4 + z_2z_3 + z_2z_4+z_3z_4 &= 0\\
z_1z_2z_3 + z_1z_2z_4+z_1z_3z_4+z_2z_3z_4 &= 0\\
z_1z_2z_3z_4 &\neq0
\end{array}\right
\end{split}\]
而
\begin{gather*}
z_1z_2z_3 + z_1z_2z_4+z_1z_3z_4+z_2z_3z_4=z_1z_2z_3z_4(\frac{1}{z_1}+\frac{1}{z_2}+\frac{1}{z_3}+\frac{1}{z_4})\\
z_k\overline{z_k} = |z_k|^2 = r^2
\end{gather*}
所以
\begin{gather*}
    \frac{1}{z_k} = \frac{\overline{z_k}}{r^2}\\
    z_1z_2z_3 + z_1z_2z_4+z_1z_3z_4+z_2z_3z_4 = \frac{z_1z_2z_3z_4}{r^2}(\overline{z_1}+\overline{z_2}+\overline{z_3}+\overline{z_4})=0
\end{gather*}
所以$z_1$,$ z_2$,$z_3$,$z_4$需满足
\[z_1z_2 + z_1z_3 + z_1z_4 + z_2z_3 + z_2z_4+z_3z_4 = 0\]
\end{homeworkProblem}
\begin{homeworkProblem}
$f(z)$在$z_0$连续,$f(z_0)\neq0$。求证$\exists\delta>0$,当$|z-z_0|<\delta$时有$f(z)\neq0$\\
\begin{proof}
因$f(z_0)\neq0$,有$|f(z_0)|>0$\\
令$\varepsilon = \frac{1}{2}|f(z_0)| < |f(z_0)|$,因为$f(z)$在$z_0$连续,存在$\delta>0$使得$\forall z\quad|z-z_0|<\delta$
\[||f(z)|-|f(z_0)|| \leq |f(z) - f(z_0)| < \frac{1}{2}|f(z_0)|\]
即
\[\frac{1}{2}|f(z_0)|<|f(z)|<\frac{3}{2}|f(z_0)|\]
\end{proof}
\end{homeworkProblem}
\newpage

\section{Chapter 2 Analytic Function}
\begin{theorem*}{Cauchy-Riemann定理}
$f(z)=u(x,y)+iv(x,y),\quad u,v\in\mathbb{C}^{(1)}$,则$f(z)$在$z_0=x_0+iy_0$点可导等价于
\[
\left\{\begin{array}{ll}
\frac{\partial{u}}{\partial{x}} &= \frac{\partial{v}}{\partial{y}} \\
\frac{\partial{u}}{\partial{y}} &= -\frac{\partial{v}}{\partial{x}}
\end{array}\right
\]
在$z_0=x_0+iy_0$点成立
\end{theorem*}
\begin{homeworkProblem}
$f(z)=f(x+iy)=u(x,y)+iv(x,y)$且$u,v\in C^{(n)}$,求$f(z)~$n阶可导的Cauchy-Riemann条件和$f^{(n)}(z)$\\
\solution
设$f'(z)=A+iB$,则
\[\begin{array}{ll}
\mathrm{d}f=f'(z)\mathrm{d}z=f'(z)(\dx+i\mathrm{d}y)
\Leftrightarrow \mathrm{d}f&= \mathrm{d}u + i\mathrm{d}v \\
&=\frac{\partial{u}}{\partial{x}}\dx+\frac{\partial{u}}{\partial{y}}\mathrm{d}y + i(\frac{\partial{v}}{\partial{x}}\dx + \frac{\partial{u}}{\partial{y}}\mathrm{d}y) \\
& = (A\dx-B\mathrm{d}y) + i(B\dx + A\mathrm{d}y)\\
\end{array}\]
由上式得
\[
\left\{\begin{array}{l}
A\dx-B\mathrm{d}y=\frac{\partial{u}}{\partial{x}}\dx+\frac{\partial{u}}{\partial{y}}\mathrm{d}y\\
B\dx + A\mathrm{d}y=\frac{\partial{v}}{\partial{x}}\dx + \frac{\partial{u}}{\partial{y}}\mathrm{d}y
\end{array}\right\]
解得
\[
\left\{\begin{array}{l}
\frac{\partial{u}}{\partial{x}}=\frac{\partial{v}}{\partial{y}}=A\\
\frac{\partial{u}}{\partial{y}}=-\frac{\partial{v}}{\partial{x}}=-B\end{array}\right
\]
即
\[f'(z)=\frac{\partial{u}}{\partial{x}} + i\frac{\partial{v}}{\partial{x}} = F'(z)\]
而
\[F'(z) = \frac{\partial{U}}{\partial{x}} + i\frac{\partial{V}}{\partial{x}}
= \frac{\partial^2u}{\partial x^2} + i\frac{\partial^2v}{\partial x^2} \]
由归纳法可证明
\[f^{(n)}(z) = \frac{\partial^nu}{\partial x^n} + i\frac{\partial^nv}{\partial x^n}\]
$u,v$需要满足Cauchy-Riemann条件
\[
\left\{\begin{array}{l}
\frac{\partial{u}}{\partial{x}}=\frac{\partial{v}}{\partial{y}}\\
\frac{\partial{u}}{\partial{y}}=-\frac{\partial{v}}{\partial{x}}
\end{array}\right\]
\end{homeworkProblem}

\begin{homeworkProblem}
    求$\cos(x+iy)$的实部和虚部,其中$x,y\in\mathbb{R}$\\
\solution
\[\begin{split}
\cos(x+iy)
&= \frac{1}{2}(e^{-y+ix} + e^{y-ix})\\
&= \frac{1}{2}(e^{-y}e^{ix} + e^ye^{-ix})\\
&= \frac{1}{2}[e^{-y}(\cos x+i\sin x) + e^y(\cos x - i\sin x)]\\
&= \frac{1}{2}(e^y + e^{-y})\cos x + i\frac{1}{2}(-e^y + e^{-y})\sin x
\end{split}\]
\end{homeworkProblem}

\begin{homeworkProblem}
    求证:$\forall A,B\in\mathbb{R}$存在$z=x+iy$使得$\cos(x+iy) = A+iB$(即$\mathrm{Im}[\cos(z)]=\mathbb{C}$)\\
\begin{proof}
令
\begin{equation}
\begin{gathered}
    \frac{e^y+e^{-y}}{2}\cos x = A\\
    \frac{e^{-y}-e^y}{2}\sin x = B
\end{gathered}
\label{eq:1}
\end{equation}
\begin{enumerate}
    \item 当$B=0$时,由式~\eqref{eq:1}~知$y=0$或$\sin x=0$。\\ $|A|\leq1$时可令$y=0$,此时
        \[\cos x = A\]
    解得\[
        \left\{\begin{array}{l}
        x=\arccos A + 2k\pi\qquad k\in\mathrm{Z}\\
        y=0
        \end{array}\right\]
    $|A|>1$时,令$\sin x=0$得
    \begin{gather*}
        \cos x = \pm1\\
        \frac{e^y+e^{-y}}{2} = |A| > 1
    \end{gather*}
    考察函数$f(y)={e^y+e^{-y}}{2}-1$
    \begin{gather*}
        f(0) = 0\\
        \lim_{y\rightarrow+\infty}f(y) = \lim_{y\rightarrow-\infty}f(y) = +\infty
    \end{gather*}
    且$f(y)$连续。因此存在$y_A$使得$\pm y_A$是方程$\frac{e^y+e^{-y}}{2} = |A|$的解。\\
    此时
    \[
    \left\{\begin{array}{l}
    x=k\pi\qquad k\in\mathrm{Z}\\
    y=\pm y_A
    \end{array}\right\]

    \item 当$B\neq0$时,由式~\eqref{eq:1}~知$y\neq0$。结合$\cos^2x+\sin^2x=1$得$y\in(-\infty,0)\cup(0,+\infty)$时
    \[
    \frac{4A^2}{(e^{-y}+e^y)^2} + \frac{4B^2}{(e^{-y}-e^y)^2} = 1
    \]
    令$f_{A,B}(y) = {4A^2}{(e^{-y}+e^y)^2} + \frac{4B^2}{(e^{-y}-e^y)^2}$,$f_{A,B}(y)$是偶函数。
    \begin{gather*}
        \lim_{y\rightarrow0^{+}}f_{A,B}(y) = +\infty\\
        \lim_{y\rightarrow+\infty}f_{A,B}(y) = 0
    \end{gather*}
    因此$\exists y_{A,B} > 0$,使得$\pm y_{A_B}$是方程
    \[
        \frac{4A^2}{(e^{-y}+e^y)^2} + \frac{4B^2}{(e^{-y}-e^y)^2} = 1
    \]
    的根。将$\pm y_{A,B}$代入式~\eqref{eq:1}~可解出对应的$x$。
\end{enumerate}
\end{proof}
\end{homeworkProblem}
\begin{homeworkProblem}
    已知$e^w=z\neq0$,求\[w=\Ln z\]\\
\solution
设$w=u+iv\quad u,v\in \mathbb{R}$,则
\begin{gather*}
e^w = e^{u+iv} = e^ue^{iv} = z = re^{i\theta} \\
\theta = \arg z \in [0,2\pi)\qquad r = |z| > 0
\end{gather*}
则
\[\begin{array}{lll}
&e^u &= r \\
\Rightarrow& u &= \ln r
\end{array}\]
且
\[\begin{array}{lll}
&e^{iv} &= e^{i\theta} \\
\Rightarrow& v &= \theta + 2k\pi \qquad k\in\mathbb{Z}\\
&&= \arg z
\end{array}\]
所以
\[\begin{array}{lll}
w = u+iv &= \Ln z  &\\
&= \ln z + 2k\pi i & k\in \mathbb{Z} \\
&= \ln|z| + i\arg z + 2k\pi i & k\in \mathbb{Z}
\end{array}\]
\begin{theorem*}{Picard 小定理}
若$f(z)$是解析函数且$f(z)$不是常数,则除去最多一个例外$w_0$,方程$f(z)=A+iB=w$至少有一个解$z$。
\end{theorem*}
\end{homeworkProblem}

\begin{homeworkProblem}
求\[\Ln (3+2i)\]
\solution
\[\begin{array} {lll}
\Ln (3+2i) &= \ln(3+2i) + 2k\pi & k\in\mathbb{Z}\\
&= \ln13 + i\arg(3+2i) + 2k\pi & k\in\mathbb{Z}
\end{array}\end{split}\]
\end{homeworkProblem}

\begin{homeworkProblem}
    求\[\Ln z^n\]
\solution
\[\begin{array} {lll}
\Ln z^n &= \ln z^n + 2k\pi & k\in\mathbb{Z}\\
&= \ln|z^n| + i\arg z^n + 2k\pi & k\in\mathbb{Z} \\
&= n\ln|z| + ni\arg z + 2k\pi & k\in\mathbb{Z} \\
&= n\Ln z
\end{array}\]
\end{homeworkProblem}

\begin{homeworkProblem}
求
\[i^{\sqrt3i}\]
\solution
\[\begin{split}
i^{\sqrt3i} &=
e^{\sqrt3i\Ln i}\\
&= e^{\sqrt3i(\frac{\pi}{2}i + 2k\pi i)} \\
&= e^{-\sqrt3(\frac{1}{2}+2k)\pi}\qquad k\in\mathbb{Z}
\end{split}\]
\end{homeworkProblem}
\newpage

\section{Chapter 3 Complex Integral}
\begin{theorem*}{Cauchy-Goursat定理}
    若$C$分段光滑,且$f(z)$在$C$上连续,在$C$内处处可导,则$\oint_C f(z)\mathrm{d}z=0$
\end{theorem*}
\begin{theorem*}{Cauchy 高阶导数公式}
\[
f^{(n)}(z_0) = \frac{n!}{2\pi i}\oint_{C_r} \frac{f(z)}{(z-z_0)^{n+1}}\mathrm{d}z
\]
这里$C_r = |z-z_0| = r$
\end{theorem*}
\begin{theorem*}{Lioville 定理}
    有界的解析函数是常数
\end{theorem*}
\begin{homeworkProblem}
求证,$f(z)$是解析函数则
\[f(z)\textrm{的像是}
\left\{\begin{array}{ll}
\textrm{二维区域} & f(z) \not\equiv C\\
\textrm{点} & f(z) \equiv C
\end{array}
\right\]
\solution
有
\begin{gather*}
    J_{(x,y)} = \left(\begin{array}{cc}
    \frac{\partial u}{\partial x} & \frac{\partial u}{\partial y} \\
    \frac{\partial v}{\partial x} & \frac{\partial v}{\partial y}
\end{array}\right)_{(x,y)} \\
    \frac{\Delta u\Delta v}{\Delta x \Delta y} = |detJ|_{(x,y)}
    = |\frac{\partial u}{\partial x} \frac{\partial v}{\partial y} - \frac{\partial u}{\partial y}\frac{\partial v}{\partial x}|_{(x,y)}
\end{gather*}
因为$f(z)=u+iv$是解析函数
\begin{gather*}
    \frac{\partial u}{\partial x}=\frac{\partial v}{\partial y}\\
    \frac{\partial u}{\partial y} = -\frac{\partial v}{\partial x}
\end{gather*}
所以
\[
\frac{\Delta u\Delta v}{\Delta x \Delta y} = (\frac{\partial u}{\partial x})^2 + (\frac{\partial v}{\partial x})^2 = |f'(x)|^2 \geq 0
\]
即
\[\Delta u\Delta v = |f'(x)|^2\Delta x \Delta y\]
若$f'(x)\not\equiv0$,那么$f(x)$不是常数。此时假设$f'(z_0)\neq0$,则
\[\exists \delta > 0\textrm{使得}|z-z_0| < \delta\textrm{时} f'(z) \neq 0\]
$|z-z_0| < \delta$时
\[\Delta u\Delta v > 0\]
即像是二维区域。\newline
当$f'(z)\equiv 0$时,$f(z)$是常数,这时$f(z)$的像是一个点

\end{homeworkProblem}

\section{Chapter 4 Series}
\begin{definition*}{幂级数}
    \[\sum_{n=0}^{+\infty}C_n(z-z_0)^n\]
\end{definition*}
\begin{definition*}{Fourier级数}
    \[\sum_{n=0}^{+\infty}C_ne^{in\theta} = \sum_{n=0}^{+\infty}(a_n\cos n\theta + b_n\sin n\theta)\]
\end{definition*}
\begin{definition*}{Taylor级数}
    \[\sum_{n=0}^{+\infty}\frac{f^{(n)}(z_0)}{n!}(z-z_0)^n\]
\end{definition*}
\begin{definition*}{Laurent级数}
    \[\sum_{n=-\infty}^{+\infty}C_n(z-z_0)^n\]
\end{definition*}
\begin{theorem*}{Abel 定理}
    若$f(z)=\sum_{n=0}^{+\infty}C_nz^n$在$z_0$收敛,则$\forall z$有$|z|<|z_0|$时$f(z)$绝对收敛。
    若存在$z_0$,$f(z)$在$z_0$发散,则$\forall z$有$|z|>|z_0|$时$f(z)$发散。(即幂级数的收敛域是圆盘)
\end{theorem*}
\begin{definition*}{收敛半径}
    若存在常数$R>0$,当$|z|<R$时,$f(z)$绝对收敛,而当$|z|>R$时,$f(z)$发散,这时$R$称为$f(z)$的收敛半径。
\end{definition*}
\begin{theorem*}
    若\[\lim_{n\rightarrow+\infty}\left|\frac{C_n}{C_{n+1}}\right|=\lambda\]
    则$R=\lambda$
\end{theorem*}
\begin{theorem*}
    若\[\lim_{n\rightarrow+\infty} \left|\frac{1}{\sqrt[n]{C_n}}\right|=\lambda\]
    则$R=\lambda$
\end{theorem*}
\begin{theorem*}
    若$f(z)$只有有限个奇点,则离原点最近的奇点$z_0$的模即为收敛半径。
\end{theorem*}
\begin{theorem*}
    若$f(z)$在$z_0$处条件收敛,则$R=|z_0|$
\end{theorem*}
\begin{theorem*}
    若$f(z)=\sum_{n=0}^{+\infty}C_nz^n$满足$C_n = a_n + ib_n\qquad a_n,b_n\in\mathbb{R}$且$\sum_{n=0}^{+\infty}a_nz^n$的收敛半径是$R_1$,$\sum_{n=0}^{+\infty}b_nz^n$的收敛半径是$R_2$,则$R=\mathrm{min}\{R_1,R_2\}$
\end{theorem*}
\begin{theorem*}
当$|z|<R$时,\[f(z)=\sum_{n=0}^{+\infty}\frac{f^{(n)}(0)}{n!}z^n\]
即在收敛圆内,$f(z)$处处满足Cauchy-Riemann条件。根据Abel定理,收敛圆上处处是奇点。
\end{theorem*}
\begin{homeworkProblem}
举出级数在其收敛圆上处处发散、既有发散的点也有收敛的点、处处收敛的例子。\newline
\solution
\begin{enumerate}
    \item 考察
    \[f(z) = \sum_{n=0}^{+\infty}z^n = \frac{1}{1-z}\qquad R=1\]
    $\forall z, |z|=1$,$f(z)$不存在,即收敛圆上处处发散。
    \item 考察
    \[f(z) = \sum_{n=1}^{+\infty}\frac{z^n}{n} \qquad R=1\]
    $f(-1)=-\ln2$但$f(1)=+\infty$发散。更一般的,对$z=e^{i\theta}\quad\theta\in[0,2\pi)$有
    \[\begin{split}
    f(e^{i\theta}) &= \sum_{n=1}^{+\infty}\frac{\cos n\theta}{n}+i\sum_{n=1}^{+\infty}\frac{\sin n\theta}{n} \\
    &= \frac{1}{2}\ln\frac{1}{2(1-\cos\theta)}+i\frac{\pi-\theta}{2}
    \end{split}
    \]
    即收敛圆上除$z=1$外都收敛。
    \item 考察
    \[
    f(z) = \sum_{n=1}^{+\infty}\frac{z^n}{n^2}\qquad R=1
    \]
    因为
    \[
    \sum_{n=1}^{+\infty}\left|\frac{z^n}{n^2}\right| \leq \sum_{n=1}^{+\infty} \frac{1}{n^2} = \frac{\pi^2}{6} < +\infty
    \]
    所以
    \[\forall z:|z|\leq1\]有$f(z)$绝对收敛
\end{enumerate}
\end{homeworkProblem}
\begin{homeworkProblem}
    将$\frac{1}{z-b}$在$z_0=a$处展成Laurent级数,$a\neq b$\\
\solution
\[\begin{split}
\frac{1}{z-b}
&= \frac{1}{-(b-a) + (z-a)}\\
&= \frac{1}{a-b}\frac{1}{1-\frac{z-a}{b-a}} \\
&= \frac{1}{a-b}\sum_{n=0}^{+\infty}\frac{(z-a)^n}{(b-a)^n}\\
&= \sum_{n=0}^{+\infty}\frac{(-1)(z-a)^n}{(b-a)^{n+1}}
\end{split}\]
条件
\[
|\frac{z-a}{b-a}| < 1
\]
即\[0\leq|z-a|<|b-a|\]
\end{homeworkProblem}
\begin{homeworkProblem}
    求\[f(z)=\frac{1}{(z-1)(z-2)}\]的Laurent级数\newline
\solution
\begin{enumerate}
    \item 当$0<|z-1|<1$时
    \[\begin{split}
    f(z) &= \frac{(z-1) - (z-2)}{(z-1)(z-2)}\\
    &= \frac{1}{z-2} - \frac{1}{z-1} \\
    &= \frac{1}{z-1 -1} - \frac{1}{z-1} \\
    &= -\frac{1}{1-(z-1)}- \frac{1}{z-1}\\
    &= -\sum_{n=0}^{+\infty}(z-1)^n - \frac{1}{z-1}\\
    &= \sum_{n=-1}^{+\infty}-(z-1)^n \qquad 0<|z-1|<1
    \end{split}\]
    \item 当$|z-1| > 1$时
    \[\begin{split}
    f(z) &= \frac{1}{z-2} - \frac{1}{z-1}\\
    &= \frac{1}{z-1 - 1}- \frac{1}{z-1}\\
    &= \frac{1}{z-1}\frac{1}{1-\frac{1}{z-1}}- \frac{1}{z-1}\\
    &= \frac{1}{z-1}\sum_{n=0}^{+\infty}(\frac{1}{z-1})^n - \frac{1}{z-1}\\
    &=\sum_{n=0}^{+\infty}(\frac{1}{z-1})^{n+1} - \frac{1}{z-1}\\
    &=\sum_{n=1}^{+\infty}(\frac{1}{z-1})^{n+1}\\
    &= \sum_{n=-\infty}^{-2}(\frac{1}{z-1})^{n}
    \end{split}\]
    \item 在$z=2$处展开同理
\end{enumerate}
\end{homeworkProblem}
\newpage

\section{Chapter 5 Residues}
\begin{definition*}
    对函数$f(z)$,若$f(z)$在$C$上连续,在$C$内有n个奇点$z_1,z_2,\cdots,z^n$。设$f(z)$在$z_k$附近可以展成Laurent级数\[f(z)=\sum_{n=-\infty}^{+\infty}C_n^{(k)}(z-z_k)^n\]
    则\[\oint_Cf(z)\mathrm{d}z = 2\pi i\sum_{k=1}^{n}C_{-1}^{(k)}\]
    称$C_{-1}^{(k)}$为$f(z)$在$z_k$点的留数,记作$\Res[f,z_k]$。即
    \[\oint_Cf(z)\mathrm{d}z = 2\pi i\sum_{k=1}^{n}\Res[f, z_k]}\]
\end{definition*}
\begin{theorem*}
    若$z_0$是$f$的一个一阶极点,且$f(z)=\frac{P(z)}{Q(z)}$,其中$P(z)$与$Q(z)$在$z_0$点解析,$P(z_0)\neq0,Q(z_0)=0,Q'(z_0)\neq0$,则$\Res[f,z_0]=\frac{P(z_0)}{Q'(z_0)}$
\end{theorem*}
\begin{homeworkProblem}
    求证
    \[
        \oint_{\vert z \vert = r > 1} \frac{1}{1+z^n}{\mathrm{d}z}
        = \oint_{\vert z \vert = r > 1} \frac{z^{2n}}{1+z^n}{\mathrm{d}z}
        = \left\{ \begin{array}{ll}
        2\pi i & n = 1 \\
        0 & n \ge 2 \end{array} \right
    \]
\textbf{\large Solution 1}
    \[
        \begin{equation*}
            \begin{split}
                \oint_{\vert z \vert = r > 1} \frac{1}{1+z^n}{\mathrm{d}z}
                & = 2\pi i \sum_{k=1}^{n}\Res [\frac{1}{1+z^n}, z_k] \\
                & = 2\pi i \sum_{k=1}^{n}\frac{1}{nz_{k}^{n-1}} \\
                & = \frac{2\pi i}{n}\sum_{k=1}^{n}\frac{1}{z_{k}^{n-1}}
            \end{split}
        \end{equation*}
    \]
    注意到 \[ z_k^n = 1 \] 则 \[ -z_k = \frac{1}{z_k^{n-1}} \]
    因此
    \[
    \begin{split}
        &\oint_{\vert z \vert = r > 1} \frac{1}{1+z^n}{\mathrm{d}z}\\
        =& \frac{2\pi i}{n}\sum_{k=1}^{n}\frac{1}{z_{k}^{n-1}}\\
        =& -\frac{2\pi i}{n}\sum_{k=1}^{n}z_k
    \end{split}
    \]
    若 $n = 1$,即 \[1 + z^n = 1 + z= 0\] 解得 \[z=-1\] 因此
    \[\oint_{\vert z \vert = r > 1} \frac{1}{1+z^n}{\mathrm{d}z} = 2\pi i\]
    否则 \[\sum_{k=1}^n z_k = 0\]
    即 \[\oint_{\vert z \vert = r > 1} \frac{1}{1+z^n}{\mathrm{d}z} = 0\]
    又
    \[\begin{split}
    \oint_{\vert z \vert = r > 1}\frac{z^{2n}}{1+z^n}\mathrm{d}z
    & = \oint_{\vert z \vert = r > 1}\frac{(z^{2n} - 1) + 1}{1+z^n} \mathrm{d}z\\
    & = \oint_{\vert z \vert = r > 1}\frac{(z^{n} - 1)(z^{n} + 1)}{1+z^n}\mathrm{d}z + \oint_{\vert z \vert = r > 1}\frac{1}{1+z^n}\mathrm{d}z \\
    & = \oint_{\vert z \vert = r > 1}(z^{n} - 1)\mathrm{d}z + \oint_{\vert z \vert = r > 1}\frac{1}{1+z^n}\mathrm{d}z
    \end{split}\]
    由Cauchy-Goursat定理 \[\oint_{\vert z \vert = r > 1}(z^{n} - 1)\mathrm{d}z  = 0\]
    所以 \[\oint_{\vert z \vert = r > 1} \frac{1}{1+z^n}{\mathrm{d}z}
    = \oint_{\vert z \vert = r > 1} \frac{z^{2n}}{1+z^n}{\mathrm{d}z}\]
    综上,
    \[
    \oint_{\vert z \vert = r > 1} \frac{1}{1+z^n}{\mathrm{d}z}
    = \oint_{\vert z \vert = r > 1} \frac{z^{2n}}{1+z^n}{\mathrm{d}z}
    = \left\{ \begin{array}{ll}
    2\pi i & n = 1 \\
    0 & n \ge 2 \end{array} \right
    \]
\textbf{\large Solution 2}\newline
令$z=re^{i\theta}\quad 0\le\theta<\2\pi,t=\frac{1}{z}$,则
\begin{gather*}
    |t| = \frac{1}{r} < 1\\
    t=\frac{1}{r}e^{-i\theta}\\
    \mathrm{d}z = \mathrm{d}(\frac{1}{t})=-\frac{1}{t^2}\mathrm{d}t\qquad0\le\theta<\2\pi\quad\textrm{积分方向为顺时针}
\end{gather*}
此时原积分
\[
\begin{split}
    \oint_{\vert z \vert = r > 1} \frac{1}{1+z^n}{\mathrm{d}z}
    &= -\oint_{|t|=\frac{1}{r}<1}\frac{1}{1+\frac{1}{t^2}}(-\frac{1}{t^2})\mathrm{d}t\\
    &= \oint_{|t|=\frac{1}{r}<1}\frac{t^{n-2}}{1+t^n}\mathrm{d}t
\end{split}
\]
只有一个奇点$t=0$。因此
\[
\begin{split}
    I_n
    &= \oint_{|t|=\frac{1}{r}<1}\frac{t^{n-2}}{1+t^n}\mathrm{d}t\\
    &=\left\bigg\{\begin{array}{ll}0& n\geq2\qquad\textrm{(Cauchy-Goursat 定理)} \\
    \oint_{|t|=\frac{1}{r}<1}\frac{1}{t(t+1)}\mathrm{d}t=2\pi i & n=1
    \end{array}\right
\end{split}
\]
\end{homeworkProblem}

\begin{homeworkProblem}
    求积分
    \[
    \oint_{|z| = r > 0}\frac{1-\cos4z^3}{z^n}\mathrm{d}z \qquad n\in\mathbb{Z}
    \]

\solution
当$n\leq 0$时
\[
\oint_{|z| = r > 0}\frac{1-\cos4z^3}{z^n}\mathrm{d}z = 0
\]
当$n > 0$时
\[\begin{split}
\frac{1-\cos4z^3}{z^n}
&=z^{-n}(1-\sum_{k=0}^n(-1)^k\frac{(4z^3)^{2k}}{(2k)!})\\
&=z^{-n}(1-\sum_{k=0}^n(-1)^k\frac{4^{2k}z^{6k}}{(2k)!})\\
&=z^{-n}\sum_{k=1}^n(-1)^{k-1}\frac{4^{2k}z^{6k}}{(2k)!}\\
\end{split}\]
$\frac{1-\cos4z^3}{z^n}$在奇点$z=0$的Laurent级数$\sum_{n=-\infty}^{+\infty}C_nz^n$中,$C_{-1}$对应上式中
\[
6k-n=-1
\]
此时
\[\begin{split}
C_{-1}&=(-1)^{k-1}\frac{4^{2k}}{(2k)!}\qquad{k=\frac{n-1}{6}}\\
&=(-1)^{\frac{n-7}{6}}\frac{4^{\frac{n-1}{3}}}{(\frac{n-1}{3})!}
\end{split}\]
所以$n > 0$时
\[\begin{split}
\oint_{|z| = r > 0}\frac{1-\cos4z^3}{z^n}\mathrm{d}z
&=2\pi i\Res[\frac{1-\cos4z^3}{z^n}, 0]\\
&=2\pi i C_{-1}\\
&=2\pi i(-1)^{\frac{n-7}{6}}\frac{4^{\frac{n-1}{3}}}{(\frac{n-1}{3})!}
\end{split}\]
综上,
\[\begin{split}
\oint_{|z| = r > 0}\frac{1-\cos4z^3}{z^n}\mathrm{d}z
= \left\{\begin{array}{ll}
2\pi i(-1)^{\frac{n-7}{6}}\cfrac{4^{\frac{n-1}{3}}}{(\frac{n-1}{3})!} & n=6k+1, k\in\mathbb{N} \\
0 & n\neq 6k+1, k\in\mathbb{N}\end{array}\right
\end{split}\]
\end{homeworkProblem}

\begin{homeworkProblem}
    求积分
    \[
    \oint_{|z| = r > 1}\frac{z^3 e^\frac{1}{z}}{1+z}\mathrm{d}z
    \]
\solution
令$t=\frac{1}{z}$,则
\begin{gather*}
    |t| = \frac{1}{r} < 1\\
    t=\frac{1}{r}e^{-i\theta}\\
    \mathrm{d}z = \mathrm{d}(\frac{1}{t})=-\frac{1}{t^2}\mathrm{d}t\qquad0\le\theta<\2\pi\quad\textrm{积分方向为顺时针}
\end{gather*}
原积分
\[\begin{split}
\oint_{|z| = r > 1}\frac{z^3 e^\frac{1}{z}}{1+z}\mathrm{d}z
&= -\oint_{|t| = \frac{1}{r} < 1}\frac{e^t}{t^2(t+1)}(-\frac{1}{t^2})\mathrm{d}t \\
&= \oint_{|t| = \frac{1}{r} < 1}\frac{e^t}{t^4(t+1)}\mathrm{d}t \\
&= 2\pi i\Res[\frac{e^t}{t^4(t+1)}, 0]
\end{split}\]
又
\[\begin{split}
\frac{e^t}{t^4(t+1)} &= t^{-4}(1+t+\frac{t^2}{2!}+\frac{t^3}{3!}+\dots)(1-t+t^2-t^3+\dots)
\end{split}\]
其中,$t^{-1}$的系数为
\[-1+1-\frac{1}{2!}+\frac{1}{3!} = -\frac{1}{3}\]
因此
\[\oint_{|z| = r > 1}\frac{z^3 e^\frac{1}{z}}{1+z}\mathrm{d}z
=2\pi i\Res[\frac{e^t}{t^4(t+1)}, 0]
=2\pi i (-\frac{1}{3})
=-\frac{2}{3}\pi i\]
\end{homeworkProblem}
\begin{homeworkProblem}
    求积分
    \[
    \int_0^{2\pi}\frac{\mathrm{d}\theta}{a+b\cos\theta} \qquad a>|b|\quad a,b\in\mathbb{R}
    \]
\solution
令$z=e^{i\theta}, \theta\in[0, 2\pi)$,注意到:
\begin{gather*}
\cos\theta = \frac{e^{i\theta}+e^{-i\theta}}{2}=\frac{z+\frac{1}{z}}{2}=\frac{z^2+1}{2z}\\
\mathrm{d}\theta = \frac{\mathrm{d}z}{ie^{i\theta}}=\frac{\mathrm{d}z}{iz}\\
\end{gather*}
则原积分
\[\begin{split}
\int_0^{2\pi}\frac{\mathrm{d}\theta}{a+b\cos\theta}
&= \oint_{|z|=1}\cfrac{\frac{\mathrm{d}z}{iz}}{a+b(\frac{z^2+1}{2z})}\\
&=\frac{2}{i}\oint_{|z|=1}\frac{\mathrm{d}z}{bz^2+2az+b}
\end{split}\]
当$b=0$时,原积分
\[\int_0^{2\pi}\frac{\mathrm{d}\theta}{a+b\cos\theta}
=\int_0^{2\pi}\frac{\mathrm{d}\theta}{a}
=\frac{2\pi}{a}\]
当$b\neq0$时,原积分
\[\begin{split}\int_0^{2\pi}\frac{\mathrm{d}\theta}{a+b\cos\theta}
&=\frac{2}{i}\oint_{|z|=1}\frac{\mathrm{d}z}{bz^2+2az+b}\\
&=\frac{2}{ib}\oint_{|z|=1}\frac{\mathrm{d}z}{z^2+\frac{2a}{b}z+1}
\end{split}\]
对方程\[z^2+\frac{2a}{b}z+1=0\]其两根\[z_1z_2=1\]且
\[z_{1,2}=-\frac{a}{b}\pm\frac{\sqrt{a^2-b^2}}{b}\]
可设$b>0$,则
\[z_2 =-\frac{a}{b}\-\frac{\sqrt{a^2-b^2}}{b}<-\frac{a}{b}<-1\]
即只有一个奇点$z_1$。所以原积分
\[\begin{split}
\int_0^{2\pi}\frac{\mathrm{d}\theta}{a+b\cos\theta}
&=\frac{2}{ib}\oint_{|z|=1}\frac{\mathrm{d}z}{z^2+\frac{2a}{b}z+1}\\
&=\frac{2}{ib}2\pi i\frac{1}{2z_1+\frac{2a}{b}}\\
&=\frac{2\pi}{\sqrt{a^2-b^2}}
\end{split}\]
\end{homeworkProblem}
\begin{homeworkProblem}
    求积分
    \[
    \int_0^{2\pi}\frac{\mathrm{d}\theta}{a+b\sin\theta} \qquad a>|b|\quad a,b\in\mathbb{R}
    \]
\solution
\begin{theorem*}
    若$f(x)$在$[-1,1]$上可积,则$\int_0^{2\pi}f(\cos x)\dtheta=\int_0^{2\pi}f(\sin x)\dtheta$
\end{theorem*}
所以
\[\int_0^{2\pi}\frac{\mathrm{d}\theta}{a+b\sin\theta}=\int_0^{2\pi}\frac{\mathrm{d}\theta}{a+b\cos\theta}=\frac{2\pi}{\sqrt{a^2-b^2}}\]
\end{homeworkProblem}
\begin{homeworkProblem}
    求积分
    \[
    I_p = \int_0^{2\pi}\frac{\mathrm{d}\theta}{1-2p\cos\theta+p^2}\qquad p\in(-1,1)
    \]
\solution
令
\begin{gather*}
a=1+p^2\\
b=-2p
\end{gather*}
则\[a>b\quad a,b\in(-1,1)\]
因此
\[\begin{split}
I_p
&= \int_0^{2\pi}\frac{\mathrm{d}\theta}{1-2p\cos\theta+p^2}\\
&= \int_0^{2\pi}\frac{\mathrm{d}\theta}{a+b\cos\theta}\\
&=\frac{2\pi}{\sqrt{a^2-b^2}} \\
&=\frac{2\pi}{1-p^2}
\end{split} \]
\end{homeworkProblem}
\begin{homeworkProblem}
    求积分
    \[
    I_{A,B} = \int_0^{2\pi}\frac{\dtheta}{A^2\cos^2\theta + B^2\sin^2\theta}
    \qquad A,B\in\mathbb{R}\quadAB>0
    \]
\solution
令
\begin{gather*}
\cos^2\theta = \frac{\cos2\theta+1}{2}\\
\sin^2\theta = \frac{1-\cos2\theta}{2}
\end{gather*}
则
\[\begin{split}
I_{A,B} &= \int_0^{2\pi}\frac{\dtheta}{A^2\cos^2\theta + B^2\sin^2\theta}\\
&= \int_0^{2\pi}\cfrac{\dtheta}{A^2\cfrac{\cos2\theta+1}{2} + B^2\cfrac{1-\cos2\theta}{2}} \\
&= \int_0^{4\pi}\frac{\mathrm{d}t}{(A^2+B^2)+(A^2-B^2)\cos t}\\
&= 2\int_0^{2\pi}\frac{\mathrm{d}t}{(A^2+B^2)+(A^2-B^2)\cos t}\\
&=2\frac{2\pi}{\sqrt{(A^2+B^2)^2-(A^2-B^2)^2}}\\
&=\frac{2\pi}{AB}
\end{split}\]
\end{homeworkProblem}

\begin{homeworkProblem}
    求积分
    \[I_n=\int_0^{+\infty}\frac{1}{(x^2+a^2)(x^2+b^2)}\dx\]
\solution
$\frac{1}{(x^2+a^2)(x^2+b^2)}$是偶函数,因此
\[\begin{split}
I_n&=\int_0^{+\infty}\frac{1}{(x^2+a^2)(x^2+b^2)}\dx\\
&=\frac{1}{2}\int_{-\infty}^{+\infty}\frac{1}{(x^2+a^2)(x^2+b^2)}\\
&=\frac{1}{2}\lim_{R \rightarrow +\infty}\int_{-R}^{+R}\frac{1}{(x^2+a^2)(x^2+b^2)}
\end{split}\]
而
\[\begin{split}
&\lim_{R \rightarrow +\infty}\int_{-R}^{+R}\frac{1}{(x^2+a^2)(x^2+b^2)}
+ \int_{|z|=R, \mathrm{Im}z>0}\frac{1}{(z^2+a^2)(z^2+b^2)}\mathrm{d}z\\
=& 2\pi i(\Res[\frac{1}{(x^2+a^2)(x^2+b^2)}, i|a|] + \Res[\frac{1}{(x^2+a^2)(x^2+b^2)}, i|b|])\\
=& 2\pi i(\frac{1}{4(i|a|)^3+2i|a|(a^2+b^2)+ab}+\frac{1}{4(i|b|)^3+2i|b|(a^2+b^2)+ab})
\end{split}\]
\begin{theorem*}
    若$P_n(z),Q_m(z)$是多项式,且$\mathrm{deg}P_n = n \leq \mathrm{deg}Q_m-2=m-2$,$Q_m(z)$在实轴$z=x$上没有零点,即$Q_m(x)\neq0,\forall x\in\mathbb{R}$,则\[\lim_{R \rightarrow +\infty}\int_{|z|=R, \mathrm{Im}z>0}\frac{P_n(z)}{Q_m(z)}\mathrm{d}z=0\]
\end{theorem*}
所以
\[\begin{split}
I_n &= \frac{1}{2}2\pi i(\Res[\frac{1}{(x^2+a^2)(x^2+b^2)}, i|a|] + \Res[\frac{1}{(x^2+a^2)(x^2+b^2)}, i|b|])\\
&=\pi i(\frac{1}{4(i|a|)^3+2i|a|(a^2+b^2)}+\frac{1}{4(i|b|)^3+2i|b|(a^2+b^2)})\\
&= \frac{\pi}{2(|b|-|a|)|a||b|}
\end{split}\]
\end{homeworkProblem}

\begin{homeworkProblem}
    求积分
    \[
    I_{n} = \int_0^{+\infty}\frac{\dx}{1+x^{2n}}\qquad n\in\mathbb{N}
    \]
\solution
\[\begin{split}
I_{n} &= \int_0^{+\infty}\frac{\dx}{1+x^{2n}}\qquad n\in\mathbb{N}\\
&= \frac{1}{2}\int_{-\infty}^{+\infty}\frac{\dx}{1+x^{2n}}\qquad n\in\mathbb{N}\\
&= \lim_{R\rightarrow+\infty}\frac{1}{2}\int_{-R}^{+R}\frac{\dx}{1+x^{2n}}\qquad n\in\mathbb{N}\\
&= \frac{1}{2}2\pi i\sum_{k=0}^{i}\Res[\frac{1}{1+z^{2n}}, z_k] - \frac{1}{2}\lim_{R\rightarrow+\infty}\int_{|z|=R, \mathrm{Im}z>0}\frac{1}{1+z^{2n}}\mathrm{d}z\\
&=\frac{1}{2}2\pi i\sum_{k=0}^{i}\Res[\frac{1}{1+z^{2n}}, z_k]
\end{split}\]
其中
\[z_k^{2n}+1=0, \mathrm{Im}z_k>0\]
则
\[\begin{split}
I_n &= \int_0^{+\infty}\frac{\dx}{1+x^{2n}}\\
&= \frac{1}{2}2\pi i\sum_{k=0}^{i}\Res[\frac{1}{1+z^{2n}}, z_k] \\
&= \pi i\sum_{k=1}^{n}\Res[\frac{1}{1+z^{2n}}, z_k]\\
&= \pi i\sum_{k=1}^{n}\frac{1}{2nz_k^{2n-1}}\\
&= -\frac{\pi i}{2n}\sum_{k=1}^{n}z_k
\end{split}\]
由\[z_k^{2n}=-1=e^{\pi i}\]得\[z_k=e^\frac{\pi i+2(k-1)\pi i}{2n}=\cfrac{e^{k\pi i}{n}}{e^{\pi i}{2n}} \qquad k=1,2,\cdots,2n\]
所以
\[\begin{split}
I_n &= -\frac{\pi i}{2n}\sum_{k=0}^{i}z_k \\
&= -\frac{\pi i}{2n}\cfrac{\sum_{k=1}^{n}e^{\frac{k\pi i}{n}}}{e^{\frac{\pi i}{2n}}}\\
&= -\frac{\pi i}{2n}\cfrac{e^{\frac{\pi i}{n}}(1-e^{\pi i})}{e^{\frac{\pi i}{2n}}(1-e^{\frac{\pi i}{n}})}\\
&=-\frac{\pi i}{n}\cfrac{e^{\frac{\pi i}{2n}}}{1-e^{\frac{\pi i}{n}}}
\end{split}\]
令$\theta=\frac{\pi i}{2n}$,则
\[\begin{split}
\cfrac{e^{\frac{\pi i}{2n}}}{1-e^{\frac{\pi i}{n}}}
&= \frac{\cos\theta + i\sin\theta}{1-\cos2\theta-i\sin\2\theta}\\
&= \frac{\cos\theta + i\sin\theta}{2\sin^2\theta-2i\sin\theta\cos\theta}\\
&= \frac{1}{-2i\sin\theta}
\end{split}\]
所以
\[\begin{split}
I_n &= -\frac{\pi i}{n}\cfrac{e^{\frac{\pi i}{2n}}}{1-e^{\frac{\pi i}{n}}}\\
&= -\frac{\pi i}{n}\frac{\cos\theta + i\sin\theta}{1-\cos2\theta-i\sin\2\theta}\\
&=-\frac{\pi i}{n}\frac{1}{-2i\sin\theta}\\
&= \cfrac{\frac{\pi}{2n}}{\sin\frac{\pi}{2n}}
\end{split}\]
\end{homeworkProblem}
\begin{homeworkProblem}
    求积分
    \[
    I_{n,r} = \int_0^{+\infty}\frac{\dx}{r^{2n}+x^{2n}}\qquad n\in\mathbb{N}
    \]
\solution
\[\begin{split}
I_{n,r} &= \int_0^{+\infty}\frac{\dx}{r^{2n}+x^{2n}}\\
&= \frac{1}{r^{2n}}\int_0^{+\infty}\frac{\dx}{1+(\frac{x}{r})^{2n}}\\
&= \frac{1}{r^{2n+1}}\int_0^{+\infty}\frac{\mathrm{d}(\frac{x}{r})}{1+(\frac{x}{r})^{2n}}\\
&= \frac{1}{r^{2n+1}}I_n
\end{split}\]
\end{homeworkProblem}
\begin{homeworkProblem}
    求积分
    \[
    J_n = \int_0^{+\infty}\frac{\dx}{(1+x^2)^n} \qquad n\in\mathbb{N}
    \]
\solution
\[\begin{split}
J_n &= \int_0^{+\infty}\frac{\dx}{(1+x^2)^n}\\
&=\frac{1}{2}\int_{-\infty}^{+\infty}\frac{\dx}{(1+x^2)^n}\\
&=\lim_{R\rightarrow+\infty}\frac{1}{2}\int_{-R}^{+R}\frac{\dx}{(1+x^2)^n}\\
&=\frac{1}{2}2\pi (i\Res[\frac{1}{(1+z^2)^n}, i]
- \lim_{R\rightarrow+\infty}\int_{|z|=R, \mathrm{Im}z>0}\frac{1}{(1+z^2)^n}\mathrm{d}z)\\
&=\frac{1}{2}(2\pi i\Res[\frac{1}{(1+z^2)^n}, i])\\
&=\pi i\Res[\frac{1}{(1+z^2)^n}, i]
\end{split} \]
而
\[\begin{split}
\frac{1}{(1+z^2)^n} = \frac{1}{(z+i)^n(z-i)^n}
\end{split}\]
令$f(z)=\frac{1}{(z+i)^n}$
\[\begin{split}
\frac{1}{(1+z^2)^n} &= \frac{1}{(z+i)^n(z-i)^n}\\
&= (z-i)^{-n}\sum_{k=0}^{+\infty}\frac{f^{(k)}(i)}{k!}(z-i)^k
\end{split}\]
要求$(z-i)^{-1}$对应的系数$C_{-1}$,对应于$k=n-1$
\[\begin{split}
C_{-1} &= \frac{f^{(n-1)}(i)}{(n-1)!}\\
&= (-1)^{n-1}\frac{(2n-2)!(2i)^{-2n+1}}{[(n-1)!]^2}
\end{split}\]
因此
\[\begin{split}
J_n &= \pi iC_{-1}\\
&= \pi i(-1)^{n-1}\frac{(2n-2)!(2i)^{-2n+1}}{[(n-1)!]^2}\\
&= \frac{\pi(2n-2)!}{[(n-1)!]^22^{2n-1}}
\end{split}\]
\end{homeworkProblem}
\begin{homeworkProblem}
    求积分
    \[
    J_{n,r} = \int_0^{+\infty}\frac{\dx}{(r^2+x^2)^n} \qquad n\in\mathbb{N}
    \]
\solution
\[\begin{split}
J_{n,r} &= \int_0^{+\infty}\frac{\dx}{(r^2+x^2)^n}\\
&= \int_0^{+\infty} \frac{1}{r^{2n}}\frac{r\mathrm{d}(\frac{x}{r})}{(1+(\frac{x}{r})^2)^n}\\
&= \frac{1}{r^{2n-1}}\int_0^{+\infty}\frac{\mathrm{d}(\frac{x}{r})}{(1+(\frac{x}{r})^2)^n}\\
&= \frac{1}{r^{2n-1}}J_n
\end{split}\]
\end{homeworkProblem}
\begin{homeworkProblem}
    求积分
    \[
    I_{a,b,k} = \int_0^{+\infty}\frac{x\sin kx}{(x^2+a^2)(x^2+b^2)}\dx
    \]
\solution
设$a\neq b$
则
\[\begin{split}
I_{a,b,k} &= \int_0^{+\infty}\frac{x\sin kx}{(x^2+a^2)(x^2+b^2)}\dx\\
&= \frac{1}{2}\mathrm{Im}\int_{-\infty}^{+\infty}\frac{xe^{\i kx}}{(x^2+a^2)(x^2+b^2)}\dx
\end{split}\]
而
\[\begin{split}
\int_0^{+\infty}\frac{xe^{\i x}}{(x^2+a^2)(x^2+b^2)}\dx
&= 2\pi i\{\Res[\frac{ze^{ikz}}{(z^2+a^2)(z^2+b^2)}, ai] + \Res[\frac{ze^{ikz}}{(z^2+a^2)(z^2+b^2)}, bi]\}\\
&= 2\pi i [\frac{aie^{-ka}}{4(ai)^3+2ai(a^2+b^2)} + \frac{bie^{-kb}}{4(bi)^3+2bi(a^2+b^2)}]\\
&= \frac{\pi i}{b^2-a^2}(e^{-ka}-e^{-kb})
\end{split}\]
所以
\[\begin{split}
I_{a,b,k}
&= \frac{1}{2}\mathrm{Im}\int_{-\infty}^{+\infty}\frac{xe^{\i kx}}{(x^2+a^2)(x^2+b^2)}\dx\\
&= \frac{1}{2}\mathrm{Im}[\frac{\pi i}{b^2-a^2}(e^{-ka}-e^{-kb})]\\
&= \frac{\pi}{2(b^2-a^2)}(e^{-ka}-e^{-kb})
\end{split}\]
$a=b$时
\[\begin{split}
I_{a,b,k}
&= \lim_{a\rightarrow b}\frac{\pi}{2(b^2-a^2)}(e^{-ka}-e^{-kb}) \\
&= \frac{-k\pi e^{-kb}}{-4b} \\
&= \frac{k\pi}{4ae^{ka}} = \frac{k\pi}{4be^{kb}}
\end{split}\]
\end{homeworkProblem}
\begin{homeworkProblem}
    求积分
    \[
    I_{a,b,k} = \int_0^{+\infty}\frac{x^2\cos kx}{(x^2+a^2)(x^2+b^2)}\dx
    \]
\solution
$a\neq b$时
\[\begin{split}
I_{a,b,k} &= \int_0^{+\infty}\frac{x^2\cos kx}{(x^2+a^2)(x^2+b^2)}\dx\\
&= \frac{1}{2}\int_{-\infty}^{+\infty}\frac{x^2\cos kx}{(x^2+a^2)(x^2+b^2)}\dx\\
&= \frac{1}{2}\mathrm{Re}\int_{-\infty}^{+\infty}\frac{x^2e^{ikx}}{(x^2+a^2)(x^2+b^2)}\dx\\
&= \frac{(be^{-kb}-ae^{-ka})\pi}{2(b^2-a^2)}
\end{split}\]
当$a=b$时
\[I_{a,b,k} = \frac{(1-ka)\pi}{4ae^{ka}}= \frac{(1-kb)\pi}{4be^{kb}}\]
\end{homeworkProblem}

\end{document}
