\documentclass{ctexart}

\usepackage{fancyhdr}
\usepackage{extramarks}
\usepackage{esint}
\usepackage{amsmath}
\usepackage{amsthm}
\usepackage{amsfonts}
\usepackage{tikz}
\usepackage[plain]{algorithm}
\usepackage{algpseudocode}

\usetikzlibrary{automata,positioning}

%
% Basic Document Settings
%

\topmargin=-0.45in
\evensidemargin=0in
\oddsidemargin=0in
\textwidth=6.5in
\textheight=9.0in
\headsep=0.25in

\linespread{1.1}

\pagestyle{fancy}
\lhead{\hmwkAuthorName}
\chead{\hmwkClass\ : \hmwkTitle}
\rhead{\firstxmark}
\lfoot{\lastxmark}
\cfoot{\thepage}

\renewcommand\headrulewidth{0.4pt}
\renewcommand\footrulewidth{0.4pt}

\setlength\parindent{0pt}

%
% Create Problem Sections
%

\newcommand{\enterProblemHeader}[1]{
    \nobreak\extramarks{}{Problem \arabic{#1} continued on next page\ldots}\nobreak{}
    \nobreak\extramarks{Problem \arabic{#1} (continued)}{Problem \arabic{#1} continued on next page\ldots}\nobreak{}
}

\newcommand{\exitProblemHeader}[1]{
    \nobreak\extramarks{Problem \arabic{#1} (continued)}{Problem \arabic{#1} continued on next page\ldots}\nobreak{}
    \stepcounter{#1}
    \nobreak\extramarks{Problem \arabic{#1}}{}\nobreak{}
}

\setcounter{secnumdepth}{0}
\newcounter{partCounter}
\newcounter{homeworkProblemCounter}
\setcounter{homeworkProblemCounter}{1}
\nobreak\extramarks{Problem \arabic{homeworkProblemCounter}}{}\nobreak{}

%
% Homework Problem Environment
%
% This environment takes an optional argument. When given, it will adjust the
% problem counter. This is useful for when the problems given for your
% assignment aren't sequential. See the last 3 problems of this template for an
% example.
%
\newenvironment{homeworkProblem}[1][-1]{
    \ifnum#1>0
        \setcounter{homeworkProblemCounter}{#1}
    \fi
    \subsection{Problem \arabic{homeworkProblemCounter}}
    \setcounter{partCounter}{1}
    \enterProblemHeader{homeworkProblemCounter}
}{
    \exitProblemHeader{homeworkProblemCounter}
}

%
% Homework Details
%   - Title
%   - Due date
%   - Class
%   - Section/Time
%   - Instructor
%   - Author
%

\newcommand{\hmwkTitle}{Recap}
\newcommand{\hmwkClass}{Introduction to Complex Analysis}
\newcommand{\hmwkAuthorName}{\textbf{Xavier Yao}}

%
% Title Page
%

\title{
    \vspace{2in}
    \textmd{\textbf{\hmwkClass:\ \hmwkTitle}}\\
    \normalsize\vspace{0.1in}\small{Due\ on\ \hmwkDueDate\ at 3:10pm}\\
    \vspace{0.1in}\large{\textit{\hmwkClassInstructor\ \hmwkClassTime}}
    \vspace{3in}
}

\author{\hmwkAuthorName}
\date{}

\renewcommand{\part}[1]{\textbf{\large Part \Alph{partCounter}}\stepcounter{partCounter}\\}

%
% Various Helper Commands
%

% Useful for algorithms
\newcommand{\alg}[1]{\textsc{\bfseries \footnotesize #1}}

% For derivatives
\newcommand{\deriv}[1]{\frac{\mathrm{d}}{\mathrm{d}x} (#1)}

% For partial derivatives
\newcommand{\pderiv}[2]{\frac{\partial}{\partial #1} (#2)}

% Integral dx
\newcommand{\dx}{\mathrm{d}x}

% Alias for the Solution section header
\newcommand{\solution}{\textbf{\large Solution}\newline}

% Probability commands: Expectation, Variance, Covariance, Bias
\newcommand{\E}{\mathrm{E}}
\newcommand{\Var}{\mathrm{Var}}
\newcommand{\Cov}{\mathrm{Cov}}
\newcommand{\Bias}{\mathrm{Bias}}

% Complex analysis
\newcommand{\Res}{\textrm{Res}}
\newcommand{\dtheta}{\mathrm{d}\theta}

% Theorem
\theoremstyle{plain}
\newtheorem*{theorem*}{定理}

\begin{document}
\section{Chapter 5 Residues}
\begin{homeworkProblem}
    求证
    \[
        \oint_{\vert z \vert = r > 1} \frac{1}{1+z^n}{\mathrm{d}z}
        = \oint_{\vert z \vert = r > 1} \frac{z^{2n}}{1+z^n}{\mathrm{d}z}
        = \left\{ \begin{array}{ll}
        2\pi i & n = 1 \\
        0 & n \ge 2 \end{array} \right
    \]
\textbf{\large Solution 1}
    \[
        \begin{equation*}
            \begin{split}
                \oint_{\vert z \vert = r > 1} \frac{1}{1+z^n}{\mathrm{d}z}
                & = 2\pi i \sum_{k=1}^{n}\Res [\frac{1}{1+z^n}, z_k] \\
                & = 2\pi i \sum_{k=1}^{n}\frac{1}{nz_{k}^{n-1}} \\
                & = \frac{2\pi i}{n}\sum_{k=1}^{n}\frac{1}{z_{k}^{n-1}}
            \end{split}
        \end{equation*}
    \]
    注意到 \[ z_k^n = 1 \] 则 \[ -z_k = \frac{1}{z_k^{n-1}} \]
    因此
    \[
    \begin{split}
        &\oint_{\vert z \vert = r > 1} \frac{1}{1+z^n}{\mathrm{d}z}\\
        =& \frac{2\pi i}{n}\sum_{k=1}^{n}\frac{1}{z_{k}^{n-1}}\\
        =& -\frac{2\pi i}{n}\sum_{k=1}^{n}z_k
    \end{split}
    \]
    若 $n = 1$,即 \[1 + z^n = 1 + z= 0\] 解得 \[z=-1\] 因此
    \[\oint_{\vert z \vert = r > 1} \frac{1}{1+z^n}{\mathrm{d}z} = 2\pi i\]
    否则 \[\sum_{k=1}^n z_k = 0\]
    即 \[\oint_{\vert z \vert = r > 1} \frac{1}{1+z^n}{\mathrm{d}z} = 0\]
    又
    \[\begin{split}
    \oint_{\vert z \vert = r > 1}\frac{z^{2n}}{1+z^n}\mathrm{d}z
    & = \oint_{\vert z \vert = r > 1}\frac{(z^{2n} - 1) + 1}{1+z^n} \mathrm{d}z\\
    & = \oint_{\vert z \vert = r > 1}\frac{(z^{n} - 1)(z^{n} + 1)}{1+z^n}\mathrm{d}z + \oint_{\vert z \vert = r > 1}\frac{1}{1+z^n}\mathrm{d}z \\
    & = \oint_{\vert z \vert = r > 1}(z^{n} - 1)\mathrm{d}z + \oint_{\vert z \vert = r > 1}\frac{1}{1+z^n}\mathrm{d}z
    \end{split}\]
    由Cauchy-Goursat定理 \[\oint_{\vert z \vert = r > 1}(z^{n} - 1)\mathrm{d}z  = 0\]
    所以 \[\oint_{\vert z \vert = r > 1} \frac{1}{1+z^n}{\mathrm{d}z}
    = \oint_{\vert z \vert = r > 1} \frac{z^{2n}}{1+z^n}{\mathrm{d}z}\]
    综上,
    \[
    \oint_{\vert z \vert = r > 1} \frac{1}{1+z^n}{\mathrm{d}z}
    = \oint_{\vert z \vert = r > 1} \frac{z^{2n}}{1+z^n}{\mathrm{d}z}
    = \left\{ \begin{array}{ll}
    2\pi i & n = 1 \\
    0 & n \ge 2 \end{array} \right
    \]
\textbf{\large Solution 2}\newline
令$z=re^{i\theta}\quad 0\le\theta<\2\pi,t=\frac{1}{z}$,则
\begin{gather*}
    |t| = \frac{1}{r} < 1\\
    t=\frac{1}{r}e^{-i\theta}\\
    \mathrm{d}z = \mathrm{d}(\frac{1}{t})=-\frac{1}{t^2}\mathrm{d}t\qquad0\le\theta<\2\pi\quad\textrm{积分方向为顺时针}
\end{gather*}
此时原积分
\[
\begin{split}
    \oint_{\vert z \vert = r > 1} \frac{1}{1+z^n}{\mathrm{d}z}
    &= -\oint_{|t|=\frac{1}{r}<1}\frac{1}{1+\frac{1}{t^2}}(-\frac{1}{t^2})\mathrm{d}t\\
    &= \oint_{|t|=\frac{1}{r}<1}\frac{t^{n-2}}{1+t^n}\mathrm{d}t
\end{split}
\]
只有一个奇点$t=0$。因此
\[
\begin{split}
    I_n
    &= \oint_{|t|=\frac{1}{r}<1}\frac{t^{n-2}}{1+t^n}\mathrm{d}t\\
    &=\left\bigg\{\begin{array}{ll}0& n\geq2\qquad\textrm{(Cauchy-Goursat 定理)} \\
    \oint_{|t|=\frac{1}{r}<1}\frac{1}{t(t+1)}\mathrm{d}t=2\pi i & n=1
    \end{array}\right
\end{split}
\]
\end{homeworkProblem}

\begin{homeworkProblem}
    求积分
    \[
    \oint_{|z| = r > 0}\frac{1-\cos4z^3}{z^n}\mathrm{d}z \qquad n\in\mathbb{Z}
    \]

\solution
当$n\leq 0$时
\[
\oint_{|z| = r > 0}\frac{1-\cos4z^3}{z^n}\mathrm{d}z = 0
\]
当$n > 0$时
\[\begin{split}
\frac{1-\cos4z^3}{z^n}
&=z^{-n}(1-\sum_{k=0}^n(-1)^k\frac{(4z^3)^{2k}}{(2k)!})\\
&=z^{-n}(1-\sum_{k=0}^n(-1)^k\frac{4^{2k}z^{6k}}{(2k)!})\\
&=z^{-n}\sum_{k=1}^n(-1)^{k-1}\frac{4^{2k}z^{6k}}{(2k)!}\\
\end{split}\]
$\frac{1-\cos4z^3}{z^n}$在奇点$z=0$的Laurent级数$\sum_{n=-\infty}^{+\infty}C_nz^n$中,$C_{-1}$对应上式中
\[
6k-n=-1
\]
此时
\[\begin{split}
C_{-1}&=(-1)^{k-1}\frac{4^{2k}}{(2k)!}\qquad{k=\frac{n-1}{6}}\\
&=(-1)^{\frac{n-7}{6}}\frac{4^{\frac{n-1}{3}}}{(\frac{n-1}{3})!}
\end{split}\]
所以$n > 0$时
\[\begin{split}
\oint_{|z| = r > 0}\frac{1-\cos4z^3}{z^n}\mathrm{d}z
&=2\pi i\Res[\frac{1-\cos4z^3}{z^n}, 0]\\
&=2\pi i C_{-1}\\
&=2\pi i(-1)^{\frac{n-7}{6}}\frac{4^{\frac{n-1}{3}}}{(\frac{n-1}{3})!}
\end{split}\]
综上,
\[\begin{split}
\oint_{|z| = r > 0}\frac{1-\cos4z^3}{z^n}\mathrm{d}z
= \left\{\begin{array}{ll}
2\pi i(-1)^{\frac{n-7}{6}}\cfrac{4^{\frac{n-1}{3}}}{(\frac{n-1}{3})!} & n=6k+1, k\in\mathbb{N} \\
0 & n\neq 6k+1, k\in\mathbb{N}\end{array}\right
\end{split}\]
\end{homeworkProblem}

\begin{homeworkProblem}
    求积分
    \[
    \oint_{|z| = r > 1}\frac{z^3 e^\frac{1}{z}}{1+z}\mathrm{d}z
    \]
\solution
令$t=\frac{1}{z}$,则
\begin{gather*}
    |t| = \frac{1}{r} < 1\\
    t=\frac{1}{r}e^{-i\theta}\\
    \mathrm{d}z = \mathrm{d}(\frac{1}{t})=-\frac{1}{t^2}\mathrm{d}t\qquad0\le\theta<\2\pi\quad\textrm{积分方向为顺时针}
\end{gather*}
原积分
\[\begin{split}
\oint_{|z| = r > 1}\frac{z^3 e^\frac{1}{z}}{1+z}\mathrm{d}z
&= -\oint_{|t| = \frac{1}{r} < 1}\frac{e^t}{t^2(t+1)}(-\frac{1}{t^2})\mathrm{d}t \\
&= \oint_{|t| = \frac{1}{r} < 1}\frac{e^t}{t^4(t+1)}\mathrm{d}t \\
&= 2\pi i\Res[\frac{e^t}{t^4(t+1)}, 0]
\end{split}\]
又
\[\begin{split}
\frac{e^t}{t^4(t+1)} &= t^{-4}(1+t+\frac{t^2}{2!}+\frac{t^3}{3!}+\dots)(1-t+t^2-t^3+\dots)
\end{split}\]
其中,$t^{-1}$的系数为
\[-1+1-\frac{1}{2!}+\frac{1}{3!} = -\frac{1}{3}\]
因此
\[\oint_{|z| = r > 1}\frac{z^3 e^\frac{1}{z}}{1+z}\mathrm{d}z
=2\pi i\Res[\frac{e^t}{t^4(t+1)}, 0]
=2\pi i (-\frac{1}{3})
=-\frac{2}{3}\pi i\]
\end{homeworkProblem}
\begin{homeworkProblem}
    求积分
    \[
    \int_0^{2\pi}\frac{\mathrm{d}\theta}{a+b\cos\theta} \qquad a>|b|\quad a,b\in\mathbb{R}
    \]
\solution
令$z=e^{i\theta}, \theta\in[0, 2\pi)$,注意到:
\begin{gather*}
\cos\theta = \frac{e^{i\theta}+e^{-i\theta}}{2}=\frac{z+\frac{1}{z}}{2}=\frac{z^2+1}{2z}\\
\mathrm{d}\theta = \frac{\mathrm{d}z}{ie^{i\theta}}=\frac{\mathrm{d}z}{iz}\\
\end{gather*}
则原积分
\[\begin{split}
\int_0^{2\pi}\frac{\mathrm{d}\theta}{a+b\cos\theta}
&= \oint_{|z|=1}\cfrac{\frac{\mathrm{d}z}{iz}}{a+b(\frac{z^2+1}{2z})}\\
&=\frac{2}{i}\oint_{|z|=1}\frac{\mathrm{d}z}{bz^2+2az+b}
\end{split}\]
当$b=0$时,原积分
\[\int_0^{2\pi}\frac{\mathrm{d}\theta}{a+b\cos\theta}
=\int_0^{2\pi}\frac{\mathrm{d}\theta}{a}
=\frac{2\pi}{a}\]
当$b\neq0$时,原积分
\[\begin{split}\int_0^{2\pi}\frac{\mathrm{d}\theta}{a+b\cos\theta}
&=\frac{2}{i}\oint_{|z|=1}\frac{\mathrm{d}z}{bz^2+2az+b}\\
&=\frac{2}{ib}\oint_{|z|=1}\frac{\mathrm{d}z}{z^2+\frac{2a}{b}z+1}
\end{split}\]
对方程\[z^2+\frac{2a}{b}z+1=0\]其两根\[z_1z_2=1\]且
\[z_{1,2}=-\frac{a}{b}\pm\frac{\sqrt{a^2-b^2}}{b}\]
可设$b>0$,则
\[z_2 =-\frac{a}{b}\-\frac{\sqrt{a^2-b^2}}{b}<-\frac{a}{b}<-1\]
即只有一个奇点$z_1$。所以原积分
\[\begin{split}
\int_0^{2\pi}\frac{\mathrm{d}\theta}{a+b\cos\theta}
&=\frac{2}{ib}\oint_{|z|=1}\frac{\mathrm{d}z}{z^2+\frac{2a}{b}z+1}\\
&=\frac{2}{ib}2\pi i\frac{1}{2z_1+\frac{2a}{b}}\\
&=\frac{2\pi}{\sqrt{a^2-b^2}}
\end{split}\]
\end{homeworkProblem}
\begin{homeworkProblem}
    求积分
    \[
    \int_0^{2\pi}\frac{\mathrm{d}\theta}{a+b\sin\theta} \qquad a>|b|\quad a,b\in\mathbb{R}
    \]
\solution
\begin{theorem*}
    若$f(x)$在$[-1,1]$上可积,则$\int_0^{2\pi}f(\cos x)\dtheta=\int_0^{2\pi}f(\sin x)\dtheta$
\end{theorem*}
所以
\[\int_0^{2\pi}\frac{\mathrm{d}\theta}{a+b\sin\theta}=\int_0^{2\pi}\frac{\mathrm{d}\theta}{a+b\cos\theta}=\frac{2\pi}{\sqrt{a^2-b^2}}\]
\end{homeworkProblem}
\begin{homeworkProblem}
    求积分
    \[
    I_p = \int_0^{2\pi}\frac{\mathrm{d}\theta}{1-2p\cos\theta+p^2}\qquad p\in(-1,1)
    \]
\solution
令
\begin{gather*}
a=1+p^2\\
b=-2p
\end{gather*}
则\[a>b\quad a,b\in(-1,1)\]
因此
\[\begin{split}
I_p
&= \int_0^{2\pi}\frac{\mathrm{d}\theta}{1-2p\cos\theta+p^2}\\
&= \int_0^{2\pi}\frac{\mathrm{d}\theta}{a+b\cos\theta}\\
&=\frac{2\pi}{\sqrt{a^2-b^2}} \\
&=\frac{2\pi}{1-p^2}
\end{split} \]
\end{homeworkProblem}
\begin{homeworkProblem}
    求积分
    \[
    I_{A,B} = \int_0^{2\pi}\frac{\dtheta}{A^2\cos^2\theta + B^2\sin^2\theta}
    \qquad A,B\in\mathbb{R}\quadAB>0
    \]
\solution
令
\begin{gather*}
\cos^2\theta = \frac{\cos2\theta+1}{2}\\
\sin^2\theta = \frac{1-\cos2\theta}{2}
\end{gather*}
则
\[\begin{split}
I_{A,B} &= \int_0^{2\pi}\frac{\dtheta}{A^2\cos^2\theta + B^2\sin^2\theta}\\
&= \int_0^{2\pi}\cfrac{\dtheta}{A^2\cfrac{\cos2\theta+1}{2} + B^2\cfrac{1-\cos2\theta}{2}} \\
&= \int_0^{4\pi}\frac{\mathrm{d}t}{(A^2+B^2)+(A^2-B^2)\cos t}\\
&= 2\int_0^{2\pi}\frac{\mathrm{d}t}{(A^2+B^2)+(A^2-B^2)\cos t}\\
&=2\frac{2\pi}{\sqrt{(A^2+B^2)^2-(A^2-B^2)^2}}\\
&=\frac{2\pi}{AB}
\end{split}\]
\end{homeworkProblem}

\begin{homeworkProblem}
    求积分
    \[I_n=\int_0^{+\infty}\frac{1}{(x^2+a^2)(x^2+b^2)}\dx\]
\solution
$\frac{1}{(x^2+a^2)(x^2+b^2)}$是偶函数,因此
\[\begin{split}
I_n&=\int_0^{+\infty}\frac{1}{(x^2+a^2)(x^2+b^2)}\dx\\
&=\frac{1}{2}\int_{-\infty}^{+\infty}\frac{1}{(x^2+a^2)(x^2+b^2)}\\
&=\frac{1}{2}\lim_{R \rightarrow +\infty}\int_{-R}^{+R}\frac{1}{(x^2+a^2)(x^2+b^2)}
\end{split}\]
而
\[\begin{split}
&\lim_{R \rightarrow +\infty}\int_{-R}^{+R}\frac{1}{(x^2+a^2)(x^2+b^2)}
+ \int_{|z|=R, \mathrm{Im}z>0}\frac{1}{(z^2+a^2)(z^2+b^2)}\mathrm{d}z\\
=& 2\pi i(\Res[\frac{1}{(x^2+a^2)(x^2+b^2)}, i|a|] + \Res[\frac{1}{(x^2+a^2)(x^2+b^2)}, i|b|])\\
=& 2\pi i(\frac{1}{4(i|a|)^3+2i|a|(a^2+b^2)+ab}+\frac{1}{4(i|b|)^3+2i|b|(a^2+b^2)+ab})
\end{split}\]
\begin{theorem*}
    若$P_n(z),Q_m(z)$是多项式,且$\mathrm{deg}P_n = n \leq \mathrm{deg}Q_m-2=m-2$,$Q_m(z)$在实轴$z=x$上没有零点,即$Q_m(x)\neq0,\forall x\in\mathbb{R}$,则\[\lim_{R \rightarrow +\infty}\int_{|z|=R, \mathrm{Im}z>0}\frac{P_n(z)}{Q_m(z)}\mathrm{d}z=0\]
\end{theorem*}
所以
\[\begin{split}
I_n &= \frac{1}{2}2\pi i(\Res[\frac{1}{(x^2+a^2)(x^2+b^2)}, i|a|] + \Res[\frac{1}{(x^2+a^2)(x^2+b^2)}, i|b|])\\
&=\pi i(\frac{1}{4(i|a|)^3+2i|a|(a^2+b^2)}+\frac{1}{4(i|b|)^3+2i|b|(a^2+b^2)})\\
&= \frac{\pi}{2(|b|-|a|)|a||b|}
\end{split}\]
\end{homeworkProblem}

\begin{homeworkProblem}
    求积分
    \[
    I_{n} = \int_0^{+\infty}\frac{\dx}{1+x^{2n}}\qquad n\in\mathbb{N}
    \]
\solution
\[\begin{split}
I_{n} &= \int_0^{+\infty}\frac{\dx}{1+x^{2n}}\qquad n\in\mathbb{N}\\
&= \frac{1}{2}\int_{-\infty}^{+\infty}\frac{\dx}{1+x^{2n}}\qquad n\in\mathbb{N}\\
&= \lim_{R\rightarrow+\infty}\frac{1}{2}\int_{-R}^{+R}\frac{\dx}{1+x^{2n}}\qquad n\in\mathbb{N}\\
&= \frac{1}{2}2\pi i\sum_{k=0}^{i}\Res[\frac{1}{1+z^{2n}}, z_k] - \frac{1}{2}\lim_{R\rightarrow+\infty}\int_{|z|=R, \mathrm{Im}z>0}\frac{1}{1+z^{2n}}\mathrm{d}z\\
&=\frac{1}{2}2\pi i\sum_{k=0}^{i}\Res[\frac{1}{1+z^{2n}}, z_k]
\end{split}\]
其中
\[z_k^{2n}+1=0, \mathrm{Im}z_k>0\]
则
\[\begin{split}
I_n &= \int_0^{+\infty}\frac{\dx}{1+x^{2n}}\\
&= \frac{1}{2}2\pi i\sum_{k=0}^{i}\Res[\frac{1}{1+z^{2n}}, z_k] \\
&= \pi i\sum_{k=1}^{n}\Res[\frac{1}{1+z^{2n}}, z_k]\\
&= \pi i\sum_{k=1}^{n}\frac{1}{2nz_k^{2n-1}}\\
&= -\frac{\pi i}{2n}\sum_{k=1}^{n}z_k
\end{split}\]
由\[z_k^{2n}=-1=e^{\pi i}\]得\[z_k=e^\frac{\pi i+2(k-1)\pi i}{2n}=\cfrac{e^{k\pi i}{n}}{e^{\pi i}{2n}} \qquad k=1,2,\cdots,2n\]
所以
\[\begin{split}
I_n &= -\frac{\pi i}{2n}\sum_{k=0}^{i}z_k \\
&= -\frac{\pi i}{2n}\cfrac{\sum_{k=1}^{n}e^{\frac{k\pi i}{n}}}{e^{\frac{\pi i}{2n}}}\\
&= -\frac{\pi i}{2n}\cfrac{e^{\frac{\pi i}{n}}(1-e^{\pi i})}{e^{\frac{\pi i}{2n}}(1-e^{\frac{\pi i}{n}})}\\
&=-\frac{\pi i}{n}\cfrac{e^{\frac{\pi i}{2n}}}{1-e^{\frac{\pi i}{n}}}
\end{split}\]
令$\theta=\frac{\pi i}{2n}$,则
\[\begin{split}
\cfrac{e^{\frac{\pi i}{2n}}}{1-e^{\frac{\pi i}{n}}}
&= \frac{\cos\theta + i\sin\theta}{1-\cos2\theta-i\sin\2\theta}\\
&= \frac{\cos\theta + i\sin\theta}{2\sin^2\theta-2i\sin\theta\cos\theta}\\
&= \frac{1}{-2i\sin\theta}
\end{split}\]
所以
\[\begin{split}
I_n &= -\frac{\pi i}{n}\cfrac{e^{\frac{\pi i}{2n}}}{1-e^{\frac{\pi i}{n}}}\\
&= -\frac{\pi i}{n}\frac{\cos\theta + i\sin\theta}{1-\cos2\theta-i\sin\2\theta}\\
&=-\frac{\pi i}{n}\frac{1}{-2i\sin\theta}\\
&= \cfrac{\frac{\pi}{2n}}{\sin\frac{\pi}{2n}}
\end{split}\]
\end{homeworkProblem}
\begin{homeworkProblem}
    求积分
    \[
    I_{n,r} = \int_0^{+\infty}\frac{\dx}{r^{2n}+x^{2n}}\qquad n\in\mathbb{N}
    \]
\solution
\[\begin{split}
I_{n,r} &= \int_0^{+\infty}\frac{\dx}{r^{2n}+x^{2n}}\\
&= \frac{1}{r^{2n}}\int_0^{+\infty}\frac{\dx}{1+(\frac{x}{r})^{2n}}\\
&= \frac{1}{r^{2n+1}}\int_0^{+\infty}\frac{\mathrm{d}(\frac{x}{r})}{1+(\frac{x}{r})^{2n}}\\
&= \frac{1}{r^{2n+1}}I_n
\end{split}\]
\end{homeworkProblem}
\begin{homeworkProblem}
    求积分
    \[
    J_n = \int_0^{+\infty}\frac{\dx}{(1+x^2)^n} \qquad n\in\mathbb{N}
    \]
\solution
\[\begin{split}
J_n &= \int_0^{+\infty}\frac{\dx}{(1+x^2)^n}\\
&=\frac{1}{2}\int_{-\infty}^{+\infty}\frac{\dx}{(1+x^2)^n}\\
&=\lim_{R\rightarrow+\infty}\frac{1}{2}\int_{-R}^{+R}\frac{\dx}{(1+x^2)^n}\\
&=\frac{1}{2}2\pi (i\Res[\frac{1}{(1+z^2)^n}, i]
- \lim_{R\rightarrow+\infty}\int_{|z|=R, \mathrm{Im}z>0}\frac{1}{(1+z^2)^n}\mathrm{d}z)\\
&=\frac{1}{2}(2\pi i\Res[\frac{1}{(1+z^2)^n}, i])\\
&=\pi i\Res[\frac{1}{(1+z^2)^n}, i]
\end{split} \]
而
\[\begin{split}
\frac{1}{(1+z^2)^n} = \frac{1}{(z+i)^n(z-i)^n}
\end{split}\]
令$f(z)=\frac{1}{(z+i)^n}$
\[\begin{split}
\frac{1}{(1+z^2)^n} &= \frac{1}{(z+i)^n(z-i)^n}\\
&= (z-i)^{-n}\sum_{k=0}^{+\infty}\frac{f^{(k)}(i)}{k!}(z-i)^k
\end{split}\]
要求$(z-i)^{-1}$对应的系数$C_{-1}$,对应于$k=n-1$
\[\begin{split}
C_{-1} &= \frac{f^{(n-1)}(i)}{(n-1)!}\\
&= (-1)^{n-1}\frac{(2n-2)!(2i)^{-2n+1}}{[(n-1)!]^2}
\end{split}\]
因此
\[\begin{split}
J_n &= \pi iC_{-1}\\
&= \pi i(-1)^{n-1}\frac{(2n-2)!(2i)^{-2n+1}}{[(n-1)!]^2}\\
&= \frac{\pi(2n-2)!}{[(n-1)!]^22^{2n-1}}
\end{split}\]
\end{homeworkProblem}
\begin{homeworkProblem}
    求积分
    \[
    J_{n,r} = \int_0^{+\infty}\frac{\dx}{(r^2+x^2)^n} \qquad n\in\mathbb{N}
    \]
\solution
\[\begin{split}
J_{n,r} &= \int_0^{+\infty}\frac{\dx}{(r^2+x^2)^n}\\
&= \int_0^{+\infty} \frac{1}{r^{2n}}\frac{r\mathrm{d}(\frac{x}{r})}{(1+(\frac{x}{r})^2)^n}\\
&= \frac{1}{r^{2n-1}}\int_0^{+\infty}\frac{\mathrm{d}(\frac{x}{r})}{(1+(\frac{x}{r})^2)^n}\\
&= \frac{1}{r^{2n-1}}J_n
\end{split}\]
\end{homeworkProblem}
\begin{homeworkProblem}
    求积分
    \[
    I_{a,b,k} = \int_0^{+\infty}\frac{x\sin kx}{(x^2+a^2)(x^2+b^2)}\dx
    \]
\solution
设$a\neq b$
则
\[\begin{split}
I_{a,b,k} &= \int_0^{+\infty}\frac{x\sin kx}{(x^2+a^2)(x^2+b^2)}\dx\\
&= \frac{1}{2}\mathrm{Im}\int_{-\infty}^{+\infty}\frac{xe^{\i kx}}{(x^2+a^2)(x^2+b^2)}\dx
\end{split}\]
而
\[\begin{split}
\int_0^{+\infty}\frac{xe^{\i x}}{(x^2+a^2)(x^2+b^2)}\dx
&= 2\pi i\{\Res[\frac{ze^{ikz}}{(z^2+a^2)(z^2+b^2)}, ai] + \Res[\frac{ze^{ikz}}{(z^2+a^2)(z^2+b^2)}, bi]\}\\
&= 2\pi i [\frac{aie^{-ka}}{4(ai)^3+2ai(a^2+b^2)} + \frac{bie^{-kb}}{4(bi)^3+2bi(a^2+b^2)}]\\
&= \frac{\pi i}{b^2-a^2}(e^{-ka}-e^{-kb})
\end{split}\]
所以
\[\begin{split}
I_{a,b,k}
&= \frac{1}{2}\mathrm{Im}\int_{-\infty}^{+\infty}\frac{xe^{\i kx}}{(x^2+a^2)(x^2+b^2)}\dx\\
&= \frac{1}{2}\mathrm{Im}[\frac{\pi i}{b^2-a^2}(e^{-ka}-e^{-kb})]\\
&= \frac{\pi}{2(b^2-a^2)}(e^{-ka}-e^{-kb})
\end{split}\]
$a=b$时
\[\begin{split}
I_{a,b,k}
&= \lim_{a\rightarrow b}\frac{\pi}{2(b^2-a^2)}(e^{-ka}-e^{-kb}) \\
&= \frac{-k\pi e^{-kb}}{-4b} \\
&= \frac{k\pi}{4ae^{ka}} = \frac{k\pi}{4be^{kb}}
\end{split}\]
\end{homeworkProblem}
\begin{homeworkProblem}
    求积分
    \[
    I_{a,b,k} = \int_0^{+\infty}\frac{x^2\cos kx}{(x^2+a^2)(x^2+b^2)}\dx
    \]
\solution
$a\neq b$时
\[\begin{split}
I_{a,b,k} &= \int_0^{+\infty}\frac{x^2\cos kx}{(x^2+a^2)(x^2+b^2)}\dx\\
&= \frac{1}{2}\int_{-\infty}^{+\infty}\frac{x^2\cos kx}{(x^2+a^2)(x^2+b^2)}\dx\\
&= \frac{1}{2}\mathrm{Re}\int_{-\infty}^{+\infty}\frac{x^2e^{ikx}}{(x^2+a^2)(x^2+b^2)}\dx\\
&= \frac{(be^{-kb}-ae^{-ka})\pi}{2(b^2-a^2)}
\end{split}\]
当$a=b$时
\[I_{a,b,k} = \frac{(1-ka)\pi}{4ae^{ka}}= \frac{(1-kb)\pi}{4be^{kb}}\]
\end{homeworkProblem}

\end{document}
