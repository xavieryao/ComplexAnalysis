\section{Chapter 2 Analytic Function}
\begin{homeworkProblem}
$f(z)=f(x+iy)=u(x,y)+iv(x,y)$且$u,v\in C^{(n)}$,求$f(z)$n阶可导的Cauchy-Riemann条件和$f^{(n)}(z)$\\
\solution
设$f'(z)=A+iB$,则
\[\begin{array}{ll}
\mathrm{d}f=f'(z)\mathrm{d}z=f'(z)(\dx+i\mathrm{d}y)
\Leftrightarrow \mathrm{d}f&= \mathrm{d}u + i\mathrm{d}v \\
&=\frac{\partial{u}}{\partial{x}}\dx+\frac{\partial{u}}{\partial{y}}\mathrm{d}y + i(\frac{\partial{v}}{\partial{x}}\dx + \frac{\partial{u}}{\partial{y}}\mathrm{d}y) \\
& = (A\dx-B\mathrm{d}y) + i(B\dx + A\mathrm{d}y)\\
\end{array}\]
由上式得
\[
\left\{\begin{array}{l}
A\dx-B\mathrm{d}y=\frac{\partial{u}}{\partial{x}}\dx+\frac{\partial{u}}{\partial{y}}\mathrm{d}y\\
B\dx + A\mathrm{d}y=\frac{\partial{v}}{\partial{x}}\dx + \frac{\partial{u}}{\partial{y}}\mathrm{d}y
\end{array}\right\]
解得
\[
\left\{\begin{array}{l}
\frac{\partial{u}}{\partial{x}}=\frac{\partial{v}}{\partial{y}}=A\\
\frac{\partial{u}}{\partial{y}}=-\frac{\partial{v}}{\partial{x}}=-B\end{array}\right
\]
即
\[f'(z)=\frac{\partial{u}}{\partial{x}} + i\frac{\partial{v}}{\partial{x}} = F'(z)\]
而
\[F'(z) = \frac{\partial{U}}{\partial{x}} + i\frac{\partial{V}}{\partial{x}}
= \frac{\partial^2u}{\partial x^2} + i\frac{\partial^2v}{\partial x^2} \]
由归纳法可证明
\[f^{(n)}(z) = \frac{\partial^nu}{\partial x^n} + i\frac{\partial^nv}{\partial x^n}\]
$u,v$需要满足Cauchy-Riemann条件
\[
\left\{\begin{array}{l}
\frac{\partial{u}}{\partial{x}}=\frac{\partial{v}}{\partial{y}}\\
\frac{\partial{u}}{\partial{y}}=-\frac{\partial{v}}{\partial{x}}
\end{array}\right\]
\end{homeworkProblem}

\begin{homeworkProblem}
    求$\cos(x+iy)$的实部和虚部,其中$x,y\in\mathbb{R}$\\
\solution
\[\begin{split}
\cos(x+iy)
&= \frac{1}{2}(e^{-y+ix} + e^{y-ix})\\
&= \frac{1}{2}(e^{-y}e^{ix} + e^ye^{-ix})\\
&= \frac{1}{2}[e^{-y}(\cos x+i\sin x) + e^y(\cos x - i\sin x)]\\
&= \frac{1}{2}(e^y + e^{-y})\cos x + i\frac{1}{2}(-e^y + e^{-y})\sin x
\end{split}\]
\end{homeworkProblem}

\begin{homeworkProblem}
    求证:$\forall A,B\in\mathbb{R}$存在$z=x+iy$使得$\cos(x+iy) = A+iB$(即$\mathrm{Im}[\cos(z)]=\mathbb{C}$)\\
\begin{proof}
令
\begin{equation}
\begin{gathered}
    \frac{e^y+e^{-y}}{2}\cos x = A\\
    \frac{e^{-y}-e^y}{2}\sin x = B
\end{gathered}
\label{eq:1}
\end{equation}
\begin{enumerate}
    \item 当$B=0$时,由式~\eqref{eq:1}~知$y=0$或$\sin x=0$。\\ $|A|\leq1$时可令$y=0$,此时
        \[\cos x = A\]
    解得\[
        \left\{\begin{array}{l}
        x=\arccos A + 2k\pi\qquad k\in\mathrm{Z}\\
        y=0
        \end{array}\right\]
    $|A|>1$时,令$\sin x=0$得
    \begin{gather*}
        \cos x = \pm1\\
        \frac{e^y+e^{-y}}{2} = |A| > 1
    \end{gather*}
    考察函数$f(y)={e^y+e^{-y}}{2}-1$
    \begin{gather*}
        f(0) = 0\\
        \lim_{y\rightarrow+\infty}f(y) = \lim_{y\rightarrow-\infty}f(y) = +\infty
    \end{gather*}
    且$f(y)$连续。因此存在$y_A$使得$\pm y_A$是方程$\frac{e^y+e^{-y}}{2} = |A|$的解。\\
    此时
    \[
    \left\{\begin{array}{l}
    x=k\pi\qquad k\in\mathrm{Z}\\
    y=\pm y_A
    \end{array}\right\]

    \item 当$B\neq0$时,由式~\eqref{eq:1}~知$y\neq0$。结合$\cos^2x+\sin^2x=1$得$y\in(-\infty,0)\cup(0,+\infty)$时
    \[
    \frac{4A^2}{(e^{-y}+e^y)^2} + \frac{4B^2}{(e^{-y}-e^y)^2} = 1
    \]
    令$f_{A,B}(y) = {4A^2}{(e^{-y}+e^y)^2} + \frac{4B^2}{(e^{-y}-e^y)^2}$,$f_{A,B}(y)$是偶函数。
    \begin{gather*}
        \lim_{y\rightarrow0^{+}}f_{A,B}(y) = +\infty\\
        \lim_{y\rightarrow+\infty}f_{A,B}(y) = 0
    \end{gather*}
    因此$\exists y_{A,B} > 0$,使得$\pm y_{A_B}$是方程
    \[
        \frac{4A^2}{(e^{-y}+e^y)^2} + \frac{4B^2}{(e^{-y}-e^y)^2} = 1
    \]
    的根。将$\pm y_{A,B}$代入式~\eqref{eq:1}~可解出对应的$x$。
\end{enumerate}
\end{proof}
\end{homeworkProblem}
\begin{homeworkProblem}
    已知$e^w=z\neq0$,求\[w=\Ln z\]\\
\solution
设$w=u+iv\quad u,v\in \mathbb{R}$,则
\begin{gather*}
e^w = e^{u+iv} = e^ue^{iv} = z = re^{i\theta} \\
\theta = \arg z \in [0,2\pi)\qquad r = |z| > 0
\end{gather*}
则
\[\begin{array}{lll}
&e^u &= r \\
\Rightarrow& u &= \ln r
\end{array}\]
且
\[\begin{array}{lll}
&e^{iv} &= e^{i\theta} \\
\Rightarrow& v &= \theta + 2k\pi \qquad k\in\mathbb{Z}\\
&&= \arg z
\end{array}\]
所以
\[\begin{array}{lll}
w = u+iv &= \Ln z  &\\
&= \ln z + 2k\pi i & k\in \mathbb{Z} \\
&= \ln|z| + i\arg z + 2k\pi i & k\in \mathbb{Z}
\end{array}\]
\begin{theorem*}{Picard 小定理}
若$f(z)$是解析函数且$f(z)$不是常数,则除去最多一个例外$w_0$,方程$f(z)=A+iB=w$至少有一个解$z$。
\end{theorem*}
\end{homeworkProblem}

\begin{homeworkProblem}
求\[\Ln (3+2i)\]
\solution
\[\begin{array} {lll}
\Ln (3+2i) &= \ln(3+2i) + 2k\pi & k\in\mathbb{Z}\\
&= \ln13 + i\arg(3+2i) + 2k\pi & k\in\mathbb{Z}
\end{array}\end{split}\]
\end{homeworkProblem}

\begin{homeworkProblem}
    求\[\Ln z^n\]
\solution
\[\begin{array} {lll}
\Ln z^n &= \ln z^n + 2k\pi & k\in\mathbb{Z}\\
&= \ln|z^n| + i\arg z^n + 2k\pi & k\in\mathbb{Z} \\
&= n\ln|z| + ni\arg z + 2k\pi & k\in\mathbb{Z} \\
&= n\Ln z
\end{array}\]
\end{homeworkProblem}

\begin{homeworkProblem}
求
\[i^{\sqrt3i}\]
\solution
\[\begin{split}
i^{\sqrt3i} &=
e^{\sqrt3i\Ln i}\\
&= e^{\sqrt3i(\frac{\pi}{2}i + 2k\pi i)} \\
&= e^{-\sqrt3(\frac{1}{2}+2k)\pi}\qquad k\in\mathbb{Z}
\end{split}\]
\end{homeworkProblem}
\newpage
