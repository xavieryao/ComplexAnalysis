\section{Chapter 6 Analytic Mappings}
\begin{definition*}
    $f(z)=\frac{az+b}{cz+d}~$其中$a,b,c,d\in\mathbb{C}$且$ad-bc\neq0$
\end{definition*}
\begin{theorem*}
    分式线性映射保广义圆。(圆$\cup$直线映射到圆$\cup$直线)
\end{theorem*}
\begin{theorem*}
    分式线性映射保对称点。即对广义圆,圆心映射到圆心,圆心的对称点映射到无穷,边界映射到边界(可由最大模原理得出)
\end{theorem*}
\begin{homeworkProblem}
求分式线性映射$w=\frac{az+b}{cz+d}$将单位圆盘$|z|<1$映射到单位圆盘$|w|<1$且使$z_1$映射到$0$,这里$|z_1|<1$
\newline
\solution
\[\begin{split}
w &= \frac{az+b}{cz+d}\\
&= \frac{a(z-(-\frac{b}{a}))}{b(z-(-\frac{d}{c}))}
\end{split}\]
根据题意\[w(z_1)=0\]
根据保对称性,$z_1$的对称点$z_1'=\frac{1}{\overline{z_1}}$满足
\[w(z_1')=\infty\]
记$\frac{a}{c}=a'$则此时
\[
\begin{array}{ll}
w &= a'\frac{z-z_1}{z-\frac{1}{\overline{z_1}}} \\
&= (-\overline{z_1}a')\frac{z-z_1}{1-\overline{z_1}z} \\
&= A(\frac{z-z_1}{1-\overline{z_1}z})
\end{array}
\]
根据最大模原理,$|w|=1$时$|z|=1$,且$|z|=1$时
\[
1=\overline{z}z=|z|^2
\]
所以$|w|=1$时
\[
\begin{split}
    1 = |A|\left|\frac{z-z_1}{1-\overline{z_1}z}\right| &= |A|\left|\frac{z-z_1}{z\overline{z}-\overline{z_1}z}\right| \\
    &= \frac{|A|}{|z|}\frac{|z-z_1|}{|\overline{z}-\overline{z_1}|} \\
    &= \frac{|A|}{|z|}\\
    &= |A|
\end{split}
\]
所以
\[
A = e^{i\theta} \qquad \theta\in[0, 2\pi)
\]
所以
\[
 w = e^{i\theta}\frac{z-z_1}{1-\overline{z_1}z} \qquad \theta\in[0,2\pi),|z_1|<1
\]
\end{homeworkProblem}
\begin{corollary*}
 $z_1=0$时$w = e^{i\theta}z$ 对应逆时针旋转$\theta$角
\end{corollary*}
\begin{corollary*}{不变式}
    从单位圆盘到单位圆盘的映射满足
    \[
    \frac{|\mathrm{d}w|}{1-|w|^2} = \frac{|\mathrm{d}z|}{1-|z|^2}
    \]
\end{corollary*}
\begin{proof}
因为从单位圆盘到单位圆盘的映射
\[
w = \frac{az+b}{cz+d} = e^{i\theta}\frac{z-z_1}{1-\overline{z_1}z}
\]
所以
\begin{gather*}
a = e^{i\theta} \\
b = -z_1e^{i\theta}\\
c = -\overline{z_1} \\
d = 1
\end{gather*}
又
\[
\frac{\mathrm{d}w}{\mathrm{d}z} = \frac{ad-bc}{(cz+b)^2}
\]
所以
\[
\left|\frac{\mathrm{d}w}{\mathrm{d}z}\right| = \frac{1-|z_1|^2}{|1-\overline{z_1}z|^2} > 0
\]
而
\[
\begin{split}
1-|w|^2 &= 1-w\overline{w}\\
&= 1 - e^{i\theta}\frac{z-z_1}{1-\overline{z_1}z}e^{-i\theta}\frac{\overline{z}-\overline{z_1}}{1-z_1\overline{z}} \\
&= \frac{(1-\overline{z_1}z)(1-z_1\overline{z}) - (z-z_1)(\overline{z}-\overline{z_1})}
{(1-\overline{z_1}z)(1-z_1\overline{z})} \\
&= \frac{1-z_1\overline{z}-\overline{z_1}z+|z|^2|z_1|^2-z\overline{z}+z\overline{z_1}+z_1\overline{z}-|z_1|^2}{|z-zz_1|^2} \\
&= \frac{1-|z_1|^2-|z|^2+|zz_1|^2}
{|z-zz_1|^2} \\
&= \frac{(1-|z|^2)(1-|z_1|^2)}
{|z-zz_1|^2}\\
\end{split}
\]
所以
\[
\left|\frac{\mathrm{d}w}{\mathrm{d}z}\right|
= \frac{1-|z_1|^2}{|1-\overline{z_1}z|^2}
 = \frac{1-|z_1|^2}{|1-zz_1|^2}
\]
所以
\[
\frac{|\mathrm{d}w|}{1-|w|^2} = \frac{|\mathrm{d}z|}{1-|z|^2}
\]
\end{proof}

\begin{homeworkProblem}
求分式线性映射使$|z-z_0|<r\longrightarrow|w-w_0|<R$且使$z_1\rightarrow w_0$。这里$|z_1-z_0|<r$。
\newline
\solution
$|z-z_0|<r$到以$z_0$为圆心的单位圆的映射为
\[
z' = \frac{z-z_0}{r}
\]
以$z_0$为圆心的单位圆到以$z_1$为圆心到单位圆的映射为
\[
w' = e^{i\theta}\frac{z'-z_1'}{1-\overline{z_1'}z'}
\]
从$|w-w_0|<R$到以$z_1$为圆心到单位圆的映射为
\[
w' = \frac{w-w_0}{R}
\]
所以
\[
w' = \frac{w-w_0}{R} = e^{i\theta}\cfrac{\frac{z-z_0}{r} - \frac{z_1-z_0}{r}}
{1 - \overline{(\frac{z_1-z_0}{r})}(\frac{z_1-z_0}{r})}
\]
解得
\[
w = w_0 + e^{i\theta}Rr\frac{z-z_1}{r^2-\overline{(z_1-z_0)}(z-z_0)} \qquad \theta\in[0,2\pi)\quad|z_1-z_0|<r
\]
\end{homeworkProblem}
\begin{homeworkProblem}
求证上半复平面$\mathrm{Im}z>0$映射到单位圆盘$|w|<1$的分式线性映射为
\[
w = e^{i\theta}\frac{z-z_1}{z-\overline{z_1}}\qquad\theta\in[0,2\pi)\quad\mathrm{Im}z_1>0
\]
\begin{proof}
根据保对称性,由于$z_1$映射到$0$,所以$z_1' = \overline{z_1}$映射到$\infty$。所以
\[w=A\frac{z-z_1}{z-\overline{z_1}}\]
当$z=x\in\mathbb{R}$时,根据最大模原理\[|w(z)| = 1\]
因此
\[
\begin{split}
1 = |w| &= |A|\left|\frac{x-z_1}{x-\overline{z_1}}\right| \\
&= |A|\left|\frac{x-z_1}{\overline{x-z_1}}\right| \\
&= |A|
\end{split}
\]
即$|A|=1,~A=e^{i\theta}\quad\theta\in[0,2\pi)$
所以
\[
w = e^{i\theta}\frac{z-z_1}{z-\overline{z_1}}\qquad\theta\in[0,2\pi)\quad\mathrm{Im}z_1>0
\]
\end{proof}
\end{homeworkProblem}

\begin{homeworkProblem}
    求证当$a,~b,~c,~d\in\mathbb{R}$时,分式线性映射使得$\mathrm{Im}z>0$映射到$\mathrm{Im}w>0$的充要条件是$ad-bc>0$
\begin{proof}
    令
\begin{gather*}
z = x+iy\\
x,y\in\mathbb{R}\\
w=\frac{az+b}{cz+d}
\end{gather*}
则
\[\begin{split}
w &= u+iv \\
&= \frac{a(x+iy)+b}{c(x+iy)+d}\\
&= \frac{(ax+b)+iay}{(cx+d)+icy}\\
&= \frac{[(ax+b)+iay][(cx+d)-icy]}{[(cx+d)+icy][(cx+d)-icy]}
\end{split}\]
而
\[
v = \mathrm{Im}w = \frac{ad-bc}{(cx+d)^2+c^2y^2}
\]
$y$与$v$同号
$
\Leftrightarrow ad-bc>0
$
\end{proof}
\end{homeworkProblem}

\begin{homeworkProblem}
$L=ax+b$是与$x$轴交于$x_0$点,与$x$轴夹角为$\alpha$的直线,其中$\alpha\in[0,\pi)$,$x_0\in\mathbb{R}$。$M=\{z=u+iv|v>ux+b\}$,求将半平面$M$映到$|w-w_0|<R$的分式线性映射。\newline
\solution
将$M=\{z=u+iv|v>ux+b\}$映射到$\mathrm{z}>0$的映射为
\[
z_1 = (z-x_0)e^{-i\alpha}
\]
设半平面$M$中$z_0$点被所求分式线性映射映射到$w_0$,$z_0$被$z_1$映射到$z_1^0$\newline
将$|w-w_0|<R$映射到单位圆的映射为
\[
w_1 = \frac{w-w_0}{R}
\]
将$\mathrm{z}>0$映射到单位圆盘的映射为
\[
w_1 = e^{i\theta}\frac{z_1 - z_1^0}{z_1-\overline{z_1^0}}
\]
即
\[
\frac{w-w_0}{R} = e^{i\theta}\frac{(z-x_0)e^{-i\alpha} - (z_0-x_0)e^{-i\alpha}}
{(z-x_0)e^{-i\alpha} - \overline{(z_0-x_0)e^{-i\alpha}}}
\]
即
\[
w = w_0 + Re^{i\theta}\frac{z-z_1}{(z-x_0)-\overline{(z_1-x_0)}e^{2i\alpha}}
\]
这里$\theta\in[0,2\pi)$,$z_1\in M$。\newline
取$\theta=0$,则
\[
w = w_0 + R\frac{z-z_1}{(z-x_0)-\overline{(z_1-x_0)}e^{2i\alpha}}
\]
\end{homeworkProblem}

\begin{homeworkProblem}
    $M$是与$x$轴夹角为$\theta$,与$x$轴交于$x_1,x_2$点的条带,其中$x_1 < x_2,~0<\theta<2\pi$。求一个单值可导映射将$M$映射到单位圆盘。\newline
\solution
将$M$映射到$N = \{z|  z\in\mathbb{C},\mathrm{Im}z > 0 \land \mathrm{Im}z < h=(x_2-x_1)\sin\theta \}$且将$x_2$映射到原点的映射为
\[
z_1 = (z-x_2)e^{-i\theta}
\]
将$N$映射到$O = \{z|  z\in\mathbb{C},\mathrm{Im}z > 0 \land \mathrm{Im}z < \pi\}$的映射为
\[
z_2 = \frac{z_1\pi}{h} = \frac{(z-x_2)e^{-i\theta}\pi}{(x_2-x_1)h\sin\theta}
\]
将$O$映射到$\mathrm{Im}z>0$的映射为
\[
z_3 = e^{z_2}
\]
将$\mathrm{Im}z>0$映射到单位圆盘的一个映射为
\[\begin{split}
w &= \frac{z_3 - i}{z_3+i}\\
&= \frac{e^{z_2} - i}{e^{z_3} + i} \\
&= \cfrac{e^{\frac{(z-x_2)e^{-i\theta}\pi}{(x_2-x_1)h\sin\theta}}-i}{e^{\frac{(z-x_2)e^{-i\theta}\pi}{(x_2-x_1)h\sin\theta}}+i}
\end{split}\]
\end{homeworkProblem}

\begin{homeworkProblem}
    求区域$D = \{|z-a|>a\textrm{与}|z-b|<b\}\textrm{之间的部分,这里}0<a<b\}$到单位圆盘$|w|<1$的单值解析映射。\newline
\solution
根据分式映射的保圆性,映射
\[
z_1 = \frac{z-2a}{z}
\]
将$D$映射到$E = \{z| z\in\mathbb{C},\mathrm{Re}z > 0 \land \mathrm{Re}z < \frac{b-a}{b}\}$,且将$z=2a$映射到原点。而
\[
z_2 = iz_1\frac{\pi}{\frac{b-a}{b}} = \frac{b-i\pi}{b-a}(\frac{z-2a}{z})
\]
将$E$映射到$F = \{z|  z\in\mathbb{C},\mathrm{Im}z > 0 \land \mathrm{Im}z < \pi\}$\newline
将$F$映射到$\mathrm{Im}z>0$的映射为
\[
z_3 = e^{z_2}
\]
将$\mathrm{Im}z>0$映射到单位圆盘的一个映射为
\[
w = \frac{z_3-i}{z_3+i}=\cfrac{e^{\frac{b-i\pi}{b-a}(\frac{z-2a}{z})}-i}
{e^{\frac{b-i\pi}{b-a}(\frac{z-2a}{z})}+i}
\]
\end{homeworkProblem}
\begin{homeworkProblem}
    区域$D=\{\textrm{弦切角为}\alpha,~\textrm{弦为}AB}\textrm{的扇形,其中}0<\alpha<\pi\}$,求将$D$映射到单位圆盘$|w|<1$的单值解析映射。\newline
\solution
分式线性映射
\[
z_1 = -\frac{z-A}{z-B}
\]
将$D$映射为$E=\{z=re^{i\theta}|z\in\mathbb{C},~r>0\land 0<\theta<\alpha\}$。而映射
\[
z_2 = z_2^{\frac{\pi}{\alpha}}
\]
将$E$映射为$\mathrm{Im}z>0$。将$\mathrm{Im}z>0$映射为单位圆盘的一个映射为
\[
w = \frac{z_2-i}{z_2+i} = \frac{(-\frac{z-A}{z-B})^{\frac{\pi}{\alpha}}-i}
{(-\frac{z-A}{z-B})^{\frac{\pi}{\alpha}}+i}
\]
\end{homeworkProblem}
\begin{homeworkProblem}
求将区域$D=\{z=re^{i\theta}|z\in\mathbb{C},~r>0\land 0<\theta<\alpha\,~0<\alpha<\pi\}$映射到圆盘$\{w|~|w-w_0|<R\}$的一个单值解析映射。
\solution
映射
\[
z_1 = z^{\frac{\pi}{\alpha}}
\]
将$D$映射到上半复平面。映射
\[
w_1 = \frac{z_1-i}{z_1+i}
\]
将上半复平面映射到单位圆盘。而
\[
w_1 = \frac{w-w_0}{R}
\]
将圆盘$\{w|~|w-w_0|<R\}$映射到单位圆盘。因此
\[
w = w_0 + R\frac{z^{\frac{\pi}{\alpha}}-i}{z^{\frac{\pi}{\alpha}}+i}
\]
特例:由复平面第一象限($\alpha=\frac{\pi}{2}$)到圆盘$\{w|~|w-w_0|<R\}$的一个单值解析映射为
\[
w = w_0 + R\frac{z^2-i}{z^2+i}
\]
\end{homeworkProblem}
\newpage
