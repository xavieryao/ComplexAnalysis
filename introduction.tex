\section{Chapter 1 Introduction}
\begin{homeworkProblem}
求\[\mathrm{max}|\alpha z^n+\beta| \qquad |z|\leq r\]
\solution
\[\mathrm{max}|\alpha z^n+\beta| = |\alpha|\mathrm{max}|z^n+\frac{\beta}{\alpha}|\]
因此下面只讨论\[\mathrm{max}|z^n+\alpha| \qquad |z|\leq r\]
有
\[|z^n+\alpha| \leq |z^n| + |\alpha| \leq r^n + |\alpha|\]
取最大值$\Leftrightarrow$等号成立$\Leftrightarrow z^n\textrm{与}\alpha\textrm{同向}\Leftrightarrow z^n = \lambda\alpha\quad \lambda > 0$\\
当$\alpha=0$时
\begin{gather*}
    |z^n| = |z|^n = r^n\\
    |z| = r\\
    \mathrm{max}|\alpha z^n+\beta| = r^n \qquad z=re^{i\thata}\quad \theta\in[0,2\pi)
\end{gather*}
当$\alpha\neq 0$时\[\mathrm{max}|z^n+\alpha|=r^n+|\alpha|\]
此时
\begin{gather*}
r^n = |z^n| = |\lambda\alpha|\\
\lambda = \frac{|r|^n}{|\alpha|}\\
z^n = \lambda\alpha=\frac{|r|^n}{|\alpha|}\alpha
\end{gather*}
设$\alpha = r_0e^{i\theta_0}$
\[\begin{split}
z^n &=\frac{|r|^n}{|\alpha|}\alpha \\
&= \frac{|r|^n}{|\alpha|}r_0e^{i\theta_0} \\
&= |r|^ne^{i\theta_0}
\end{split}\]
则取最大值时\[z=z_k=re^{\frac{i(\theta_0+2k\pi)}{n}} \qquad k=1,2,\cdots,n\]
综上,
\[
\mathrm{max}|z^n+\alpha| = \left\{\begin{array}{lll}
r^n & z=re^{i\theta}\quad \theta\in[0,2\pi) & \alpha=0\\
r^n + |\alpha| & z=re^{\frac{i(\theta_0+2k\pi)}{n}} \quad k=1,2,\cdots,n & \alpha\neq0
\end{array}\right
\]
\end{homeworkProblem}

\begin{homeworkProblem}
    \[|z|^2=z\overline{z}\qquad z^2=|z|^2\qquad\]
等号成立的条件是?\newline
\solution
\[z^2 = |z|^2 = z\overline{z} \Leftrightarrow z(z-\overline{z})=0\Leftrightarrow z=\overline{z}\]
即$z\in\mathbb{R}$时等号成立。
\end{homeworkProblem}


\begin{homeworkProblem}
证明\[|z_1+ z_2|^2 + |z_1 - z_2|^2 = 2(|z_1|^2 + |z_2|^2)\] 并说明几何意义\newline
\begin{proof}
\[\begin{split}
|z_1+ z_2|^2 + |z_1 - z_2|^2
&= (z_1+z_2)\overline{(z_1+z_2)} + (z_1-z_2)\overline{(z_1-z_2)}\\
&=(z_1+z_2)(\overline{z_1}+\overline{z_2}) + (z_1-z_2)(\overline{z_1}-\overline{z_2})\\
&=z_1\overline{z_1} + z_1\overline{z_2} + z_2\overline{z_1} + z_2\overline{z_2} + z_1\overline{z_1}-z_1\overline{z_2}-z_2\overline{z_1} + z_2\overline{z_2} \\
&= 2(z_1\overline{z_1} + z_2\overline{z_2})\\
&= 2(|z_1|^2 + |z_2|^2)
\end{split}\]
\end{proof}
几何意义:平行四边形对角线平方和等于对边平方和
\end{homeworkProblem}

\begin{homeworkProblem}
    $|z_1|=|z_2|=|z_3|=|z_4|=r\textrm{且}z_1+z_2+z_3+z_4=0$则$z_1$,$ z_2$,$z_3$,$z_4$满足什么条件时$z_1z_2z_3z_4$构成正方形?\newline
\solution
\begin{theorem*}
    $z_1,z_2,\cdots,z_n$将圆$|z|=\alpha$n等分$\Leftrightarrow$$z_k$是分圆多项式$z^n+\alpha=0$
\end{theorem*}
当$z_1$,$ z_2$,$z_3$,$z_4$构成正方形时,$(z-z_1)(z-z_2)(z-z_3)(z-z_4)$是分圆多项式。\\
又
\[\begin{split}
&(z-z_1)(z-z_2)(z-z_3)(z-z_4)\\
=& z^4 - \sum_{k=1}^{4}z_kz^3 + (z_1z_2 + z_1z_3 + z_1z_4 + z_2z_3 + z_2z_4+z_3z_4) z^2 \\
&-(z_1z_2z_3 + z_1z_2z_4+z_1z_3z_4+z_2z_3z_4)z + z_1z_2z_3z_4
\end{split}\]
$(z-z_1)(z-z_2)(z-z_3)(z-z_4)${是分圆多项式}
\[\begin{split}
\Leftrightarrow& \left\{\begin{array}{ll}
\sum_{k=1}^{4}z_k &= 0\\
z_1z_2 + z_1z_3 + z_1z_4 + z_2z_3 + z_2z_4+z_3z_4 &= 0\\
z_1z_2z_3 + z_1z_2z_4+z_1z_3z_4+z_2z_3z_4 &= 0\\
z_1z_2z_3z_4 &\neq0
\end{array}\right
\end{split}\]
而
\begin{gather*}
z_1z_2z_3 + z_1z_2z_4+z_1z_3z_4+z_2z_3z_4=z_1z_2z_3z_4(\frac{1}{z_1}+\frac{1}{z_2}+\frac{1}{z_3}+\frac{1}{z_4})\\
z_k\overline{z_k} = |z_k|^2 = r^2
\end{gather*}
所以
\begin{gather*}
    \frac{1}{z_k} = \frac{\overline{z_k}}{r^2}\\
    z_1z_2z_3 + z_1z_2z_4+z_1z_3z_4+z_2z_3z_4 = \frac{z_1z_2z_3z_4}{r^2}(\overline{z_1}+\overline{z_2}+\overline{z_3}+\overline{z_4})=0
\end{gather*}
所以$z_1$,$ z_2$,$z_3$,$z_4$需满足
\[z_1z_2 + z_1z_3 + z_1z_4 + z_2z_3 + z_2z_4+z_3z_4 = 0\]
\end{homeworkProblem}
\begin{homeworkProblem}
$f(z)$在$z_0$连续,$f(z_0)\neq0$。求证$\exists\delta>0$,当$|z-z_0|<\delta$时有$f(z)\neq0$\\
\begin{proof}
因$f(z_0)\neq0$,有$|f(z_0)|>0$\\
令$\varepsilon = \frac{1}{2}|f(z_0)| < |f(z_0)|$,因为$f(z)$在$z_0$连续,存在$\delta>0$使得$\forall z\quad|z-z_0|<\delta$
\[||f(z)|-|f(z_0)|| \leq |f(z) - f(z_0)| < \frac{1}{2}|f(z_0)|\]
即
\[\frac{1}{2}|f(z_0)|<|f(z)|<\frac{3}{2}|f(z_0)|\]
\end{proof}
\end{homeworkProblem}
\newpage
